\documentclass[a4paper, 12pt]{article}
\usepackage[top=2.5cm, left=3.5cm, right=1.5cm, bottom=2.5cm, includefoot]{geometry}

% LANGUAGE %
%-========-%
\usepackage{luavlna}                            % auto non-breaking spaces
\AtBeginDocument{\singlechars{czech}{AaIiVvOoUuSsZzKk}}
\usepackage[czech]{babel}                       % use the czech language


% TOC AND LOF %
%-===========-%
\usepackage[subfigure,titles]{tocloft}          % for controlling LOF and LOF
\setlength{\cftfigindent}{0pt}                  % remove left LOF margin

\usepackage[numbib, nottoc]{tocbibind}          % number bib, no TOC in TOC

\makeatletter                                   % LOF numbering
\renewcommand\listoffigures{%
    \section{\listfigurename}%
    \@mkboth{\MakeUppercase\listfigurename}%
        {\MakeUppercase\listfigurename}%
    \@starttoc{lof}%
}
\makeatother

\renewcommand{\contentsname}{Obsah}             % change TOC name


% CODE %
%-====-%
\usepackage{minted}


% GRAPHICS %
%-========-%
\usepackage{graphicx}                           % graphics package
\graphicspath{ {./img/} }                       % graphics path

\usepackage{float}                              % floats in correct position
\usepackage{subfig}                             % subfigure package

\setlength{\fboxsep}{0pt}                       % image border separation
\setlength{\fboxrule}{1pt}                      % image border thickness


% MISCELLANEOUS %
%-=============-%
\usepackage{tabularx}                           % tables
\usepackage[hyphens,spaces,obeyspaces]{url}     % url
\usepackage[hidelinks,unicode]{hyperref}        % interactive links
\usepackage{amsmath, amstext}                   % math

\usepackage[font=small,labelfont=bf]{caption}   % caption settings
\usepackage[figurename=Obr.]{caption}           % caption name


% DATETIME %
%-========-%
\usepackage[dmyyyy]{datetime}                   % dmyyyy datetimes
\renewcommand{\dateseparator}{. }               % changes datetime separator


% FONT AND ENCODING %
%-=================-%
\usepackage[T1]{fontenc}                        % font encoding
\usepackage{fontspec}                           % font selector

\setmainfont{CMU Serif}                         % select font that suports bx sc

\setlength\parindent{0pt}                       % paragraph indentation
\setlength{\parskip}{1.2em}                     % paragraph spacing


% ACRONYMS %
%-================-%
\usepackage[acronym,nopostdot,numberedsection,nogroupskip,nonumberlist]{glossaries}
\renewcommand{\acrfullformat}[2]{#2\space(#1)}  % switch long (short) for short (long)

\makeglossaries

\newacronym{frc}{FRC}{First Robotics Competition}
\newacronym{ftc}{FTC}{First Technical Challenge}
\newacronym{fll}{FLL}{First Lego League}
\newacronym{pdf}{PDF}{Portable Document Format}
\newacronym{html}{HTML}{HyperText Mark-up Language}
\newacronym{cern}{CERN}{Conseil Européen pour la Recherche Nucléaire}
\newacronym{http}{HTTP}{HyperText Transfer Protocol}
\newacronym{css}{CSS}{Cascading Style Sheets}
\newacronym{sass}{SASS}{Syntactically Awesome Style Sheets}
\newacronym{cvs}{CVS}{Concurrent Versions System}
\newacronym{cad}{CAD}{Computer-Aided Design}
\newacronym{ftp}{FTP}{File Transfer Protocol}
\newacronym{aes}{AES}{Advanced Encryption Standard}
\newacronym{api}{API}{Application Programming Interface}
\newacronym{si}{SI}{Système international}
\newacronym{ip}{IP}{Internet Protocol}
\newacronym{png}{PNG}{Portable Network Graphics}
\newacronym{wysiwyg}{WYSIWYG}{What You See Is What You Get}


% TITLE FORMATTING %
%-================-%
\usepackage{titlesec}

% title formatting
\titleformat{\section}{\bfseries\scshape\fontsize{18}{21.6}\selectfont}{\thesection}{1em}{}
\titleformat{\subsection}{\bfseries\fontsize{16}{19.2}\selectfont}{\thesubsection}{1em}{}
\titleformat{\subsubsection}{\bfseries\fontsize{14}{16.8}\selectfont}{\thesubsubsection}{1em}{}

% title spacing
\titlespacing{\section}{0pt}{0pt}{0.3em}
\titlespacing{\subsection}{0pt}{0pt}{0.2em}
\titlespacing{\subsubsection}{0pt}{0pt}{0.0em}


% OTHER %
%-=====-%
\widowpenalty10000  % prevent widows
\clubpenalty10000   % prevent widows

\pagenumbering{gobble}   % suppress page numbering until TOC page

% define KaTeX logo
\makeatletter
\DeclareRobustCommand{\KaTeX}{%
  K\kern -.19em
  {\sbox \z@ T\vbox to\ht \z@ {\hbox{%
  \check@mathfonts
  \fontsize\sf@size\z@
  \selectfont A}%
  \vss}%
}\kern -.15em
\TeX}
\makeatother


\begin{document}
  % should fix some lines going to the margins
  \sloppy

  % first page's geometry is different
  \newgeometry{top=2.5cm, left=2.5cm, right=2.5cm, bottom=2.5cm}

  \bfseries

  \begin{center}
    {\fontsize{18}{21.6} \selectfont STŘEDOŠKOLSKÁ ODBORNÁ ČINNOST}\\%
    \vspace*{\baselineskip}
    {\fontsize{14}{16.8} \selectfont Obor č. 18: Informatika}\\%

    \topskip0pt
    \vspace{16em}
    {\fontsize{20}{24} \selectfont Robotika Jednoduše}%
    \vspace*{\fill}
  \end{center}

  \fontsize{16}{19.2} \selectfont
  Tomáš Sláma\\
  Liberecký Kraj
  \hfill
  Turnov 2019

  % the geometry for the rest of the document
  \newgeometry{top=2.5cm, left=3.5cm, right=1.5cm, bottom=2.5cm}

  \newpage
  \begin{center}
    {\fontsize{18}{21.6} \selectfont STŘEDOŠKOLSKÁ ODBORNÁ ČINNOST}\\%
    \vspace*{\baselineskip}
    {\fontsize{14}{16.8} \selectfont Obor č. 18: Informatika}\\%

    \topskip0pt
    \vspace{10em}
    \fontsize{20}{24} \selectfont
    Robotika Jednoduše%

    Robotics Simplified%
    \vspace*{\fill}
  \end{center}

  \normalfont
  \fontsize{16}{19.6} \selectfont

  \textbf{Autoři:} Tomáš Sláma\\
  \textbf{Škola:} Gymnázium, Turnov, Jana Palacha 804, příspěvková \\ organizace, 511 21 Turnov \\
  \textbf{Kraj:} Liberecký kraj \\
  \textbf{Konzultant:} Mgr. Tomáš Novotný

  \vspace{\baselineskip}

  \fontsize{12}{14.4} \selectfont
  Turnov 2019

  \vspace{4em}

  \newpage

  \section*{\normalfont\textbf{Prohlášení}}
  Prohlašuji, že jsem svou práci SOČ vypracoval/a samostatně a použil/a jsem pouze prameny a literaturu uvedené v seznamu bibliografických záznamů.

  Prohlašuji, že tištěná verze a elektronická verze soutěžní práce SOČ jsou shodné.

  Nemám závažný důvod proti zpřístupňování této práce v souladu se zákonem č. 121/2000 Sb., o právu autorském, o právech souvisejících s právem autorským a o změně některých zákonů (autorský zákon) ve znění pozdějších předpisů.

  \qquad

  V Turnově dne \today \, .......................................................\\%
  \makebox[\linewidth]{Tomáš Sláma}%

  \newpage

  \section*{\normalfont\textbf{Poděkování}}
  V první řadě bych rád poděkoval své rodině a přátelům za neustálou podporu a pozitivitu při tvorbě projektu. Za konzultace o obsahu, struktuře a použitých technologiích děkuji Lukáši Pitoňákovi, Janu Poláškovi a Tomáši Novotnému. Také děkuji Kateřině Sulkové za rady ke psaní samotné SOČ práce a na závěr všem návštěvníkům projektu, kteří svou přítomností a feedbackem pomáhají v jeho rozvoji.

  % TODO: add actual titles for

  \newpage

  \section*{\normalfont\textbf{Anotace}}
  Tato práce popisuje proces tvorby webové stránky \url{robotics-simplified.com}. Stránka zpracovává témata z oboru robotiky formou pochopitelnou i naprostým začátečníkem. Je poháněna generátorem statických stránek Jekyll, který doplňují JavaSriptové knihovny pro vykreslování matematických rovnic, analýzy návštěvnosti a interaktivní vizualizaci probíraných konceptů. Python scripty automatizují kompresi obrázků, generování souboru sitemap, nahrání obsahu na hosting přes FTP a převod stránky do knižní podoby pro offline čtení.

  \subsection*{\fontsize{18}{21.6}\selectfont\normalfont\textbf{Klíčová slova}}
  robotika; webová stránka; vzdělávání

  \section*{\normalfont\textbf{Annotation}}
  This paper decribes the process of creating the \url{robotics-simplified.com} website. The site covers topics in the field of robotics in a beginner-friendly way. It is powered by the static site generator Jekyll, and is suplemented by JavaScript libraries for rendering mathematical equations, analyzing the website traffic and creating interactive visualizations of the discussed concepts. Python scripts fully automate image compression, sitemap file generation, content upload via FTP and conversion to a book version for offline reading.

  \section*{\normalfont\textbf{Keywords}}
  robotics; website; education

  \newpage

  % start page numbering at the first TOC page
  \newcounter{savepage}
  \setcounter{savepage}{\value{page}}%
  \pagenumbering{arabic}
  \setcounter{page}{\numexpr\value{savepage}-1}%

  % genereate TOC
  \tableofcontents

  \newpage

  \section{Úvod}
  V dnešním světě plném technologií se robotika rozvíjí stále rapidnějším tempem. Roboti jsou díky nadaným programátorům a inženýrům rychlejší, obratnější a chytřejší než kdy jindy a uplatnění nalézají v nesčetném množství oborů. Jen za rok 2018 se prodej industriálních robotů zvýšil o 30 \% \cite{industrial-robot-growth} a nadnárodní korporace jako Amazon jich používají na úkor lidské pracovní síly stále více \cite{amazon-hiring}. Bez talentovaných lidí by však takový pokrok nebyl možný, proto vzniká řada programů a organizací, které se vzdělávání budoucí generace snaží podporovat.

  Značnou zásluhu má například neziskový program FIRST (\url{https://www.firstinspires.org/}) a jeho soutěže \gls{frc}, \gls{ftc} a \gls{fll}, které z robotiky dělají hru. Týmy tvořeny studenty základních a středních škol mají za úkol během několika týdnů robota od základu navrhnout, postavit a naprogramovat tak, aby plnil úkoly každým rokem se měnící výzvy.

  Začátečníci se zájmem o robotiku se však musí potýkat s problematikou nedostatku studijních materiálů, ať už se jedná o programování robota či o jeho stavbu. Čtení odborných prací však není pro každého, zvláště pokud o daném oboru ví málo, a hledání přístupnějších kvalitních zdrojů ne vždu končí úspěchem.

  Stránka \url{robotics-simplified.com} se snaží tento problém řešit tím, že poskytuje centralizovaný zdroj informací pro nováčky do oboru robotiky. Je strukturována jako série článků, které koncepty intuitivním způsobem vysvětlují. Články obsahují ilustrace a interaktivní vizualizace, a jsou pro zájemce o programování doplněny zdrojovými kódy, které koncepty implementují. Celá stránka je rovněž optimalizovaná pro mobilní zařízení a je dostupná také ve formátu \gls{pdf}, proto připojení k internetu není pro čtení stránky třeba.

  Tato práce se zabývá procesem tvorby, provozem a fungováním výše zmíněné webové stránky. Popisuje nástroje a programovací jazyky, které stránka využívá a opodstatňuje jejich použití oproti alternativám. Ke konci rozebírá statistiky návštěvnosti a jejich dělení podle kritérií jako věk, pohlaví, délka návštěvy stránky a jiné.

  \newpage

  \section{Požadavky na výsledný produkt} \label{sec:Požadavky na výsledný produkt}
  Nápad tvorby kvalitních materiálů pro studium robotiky je příliš rozsáhlý na to, aby mohla práce na projektu začít bez větší úvahy~--~bude se jednat o webovou stránku, aplikaci, knihu, či repozitář se zdrojovými kódy?

  Před začátkem je třeba určit požadavky, které by výsledný produkt měl splňovat. Podle těch lze způsoby provedení porovnat a použít ten, který se jeví za nejvhodnější.


  \subsection{Otevřenost zdrojového kódu}
  Hlavní požadavek, který by měl produkt splňovat je \emph{otevřenost} (tj. bezplatná dostupnost) zdrojového kódu. Otevřenost výrazně usnadňuje kolaboraci (kdokoliv může navrhnout změnu) a zvyšuje důvěru v daný software (chování programu lze snadno ověřit nahlednutím do kódu).

  Příklady populárního otevřeného softwaru využívaného při stavbě tohoto projektu jsou např. Inkscape, Atom, Git či Jekyll, které jsou podrobněji rozebírány v kapitole \ref{sec:Použité nástroje}.


  \subsection{Multiplatformnost}
  Co se dostupnosti týče, důležitým požadavkem je \emph{multiplatformnost}, tj. možnost přístupu k produktu z více než jedné platformy jako stolní počítač, mobil, či čtečka elektronických knih. Výsledný produkt musí být multiplatformní zejména kvůli mému osobnímu názoru, že vzdělání by nemělo být omezeno technologickými preferencemi jednotlivce.

  Tato podmínka fakticky znemožňuje tvorbu aplikace, jelikož by finanční i časové nároky stavby aplikací pro vícero systémů přerostly nad rámec jednotlivce.


  \section{Použité programovací jazyky}
  Programovací jazyk je notace pro psaní programů. \emph{Syntaxe} jazyka popisuje strukturu~--~jakým způsobem za sebe uspořádat znaky, aby v rámci jazyka tvořily platná spojení. \emph{Sémantika} jazyka popisuje význam těchto platných spojení \cite{intro-to-programming-languages}.

  Každý programovací jazyk využívá \emph{paradigmata}~--~způsoby, kterými programovací jazyk přistupuje ke tvorbě programů. Paradigmaty můžeme dělit programovací jazyky do kategorií\footnote{Jeden programovací jazyk může využívat více paradigmat a spadat tak do několika kategorií (viz. kapitoly \ref{sec:Python} a \ref{sec:JavaScript}).} jako např. funkcionální, procedurální, či logické.

  Následující kapitola rozebírá použití programovacích jazyků v projektu a jejích stručnou historii.


  \subsection{HTML} \label{sec:HTML}
  Značkovací jazyk \gls{html} vytvořil Tim Berners-Lee z výzkumného centra \gls{cern} v roce 1989. Hlavní důvod vzniku byla snaha zjednodušit přístup k odborné literatuře vkládáním odkazů přímo do samotných dokumentů. V rámci této snahy Tim rovněž vytvořil protokol \gls{http}, který slouží ke přenosu souborů \gls{html} po síti \cite{html-history}.

  % TODO: write more about how the structuring works and how there should be no styling in HTML

  V \gls{html} je psán zdrojový kód převážné většiny webových stránek včetně kódu tohoto projektu, který je generován programem Jekyll.


  \subsection{CSS} \label{sec:CSS}
  Po vzniku \gls{html} bylo třeba vytvořit stylopisný jazyk, který by umožňoval přizpůsobení vzhledu \gls{html} dokumentů. Jedním z průkopníků této myšlenky byl Håkon Wimu Lie, který r. 1994 publikoval první specifikaci stylopisného standardu \gls{css} \cite{css-proposal, css-saga}, který umožňuje ovlivnit vlastnosti \gls{html} tagů jako barva a velikost písma, textura pozadí, či pozice obrázku, které samotné \gls{html} neumožňuje\footnote{To, že v \gls{html} nejde měnit vzhled stránky není pravda, ale většinou se jedná o zastaralé a nepoužívané tagy (jako např. \texttt{<center>}), jejichž použití je oproti \gls{css} nepotřebné.}.

  Při použití \gls{css} je důležitý princip kaskádování~--~vzhled stránky mohou ovlivnit jak požadavky čtenáře, tak požadavky autora. Samotné kaskádování je kombinace těchto požadavků a případná rezoluce takto vzniklých konfliktů (autor chce písmo velikosti \texttt{12pt}, ale čtenář velikosti \texttt{10pt}).

  V projektu je používán \gls{sass}, který převádí \gls{css} s pokročilejším syntaxem a dodatečnou funkcionalitou do standardního \gls{css}, díky čemuž jde vzhled stránky upravovat jednodušeji.


  \subsection{JavaScript} \label{sec:JavaScript}
  Okolo roku 1995 bylo pro internetový prohlížeč Netscape potřeba vytvořit interpretovaný scriptovací jazyk pro zlepšení interaktivity webových stránek, který by měl jít vložit přímo do \gls{html}. Tímto úkolem byl pověřen Brendan Eich a jelikož bylo z ekonomických důvodů potřeba práci vykonat rychle, první funkční verze JavaScriptu byla údajně zhotovena za pouhých 10 dní \cite{the-origin-of-javascript}.

  Provádění JavaScriptového kódu probíhá na straně uživatele, díky čemuž není server hostující danou webovou stránku zatěžován. % TODO: add more writing about the language

  Dnes se jedná o webový standard, díky kterému přidat to webové stránky interaktivitu a o dlouhodobě nejpoužívanější programovací jazyk na platformě GitHub \cite{github-statistics} a na kterém stojí řada podstatných funkcí tohoto projektu.


  \subsubsection{p5.js} \label{sec:p5.js}
  p5.js je JavaScriptová knihovna založena na softwaru pro tvorbu interaktivních „skečů“ Processing. Umožňuje vykreslovat tvary a křivky pomocí volání jednoduchých funkcí, proto se jedná o populární volbu pro umělce, učitele, či designéry.


  \subsubsection{\texorpdfstring{\KaTeX}{KaTeX}} \label{sec:KaTeX}
  \KaTeX{} (\url{https://katex.org/}) je JavaScriptová knihovna vytvořena pro vzdělávací stránku Khan Academy, která je na stránce používána na renderování matematických rovnic z \LaTeX ové notace (viz. kapitola \ref{sec:Matematika}).

  Populární alternativa, která byla na stránce dříve používána je knihovna MathJax (\url{https://www.mathjax.org/}), která je však oproti \KaTeX u znatelně pomalejší \cite{katex-mathjax-comparison}, proto již není používána.


  \subsection{Python} \label{sec:Python}
  Python je programovací jazyk vytvořený pro distribuovaný operační systém Amoeba Guidem van Rossumem v 80. letech 20. století. Byl silně inspirován Guidovou prací na jazyce ABC: „Vzal jsem ingredience z ABC a trochu jsem je promíchal. Python je v mnoha věcech podobný ABC, ale v mnoha také rozdílný \cite{making-of-python}“ (přeloženo z angličtiny).

  Od svého vzniku se Python vyvinul v mocný a frekventovaně používaný programovací jazyk~--~má rozsáhlou standardní knihovnu, podporuje mnoho různých programovacích paradigmat a je třetím nejpoužívanějším programovacím jazykem na platformě GitHub \cite{github-statistics}, od stavby webových stránek po knihovny na trénování neuronových sítí.

  Na stránce zastává funkci scriptovacího jazyka pro automatizaci provozu (viz. kapitola \ref{sec:Automatizace provozu}).


  \subsection{Markdown} \label{sec:Markdown}
  Markdown je styl formátování prostého textu, vytvořený v roce 2004 Johnem Gruberem pro autory webových stránek. Je uzpůsobený k jednoduchému čtení, psaní, a převodu do pokročilejších značkovacích jazyků jako \gls{html} \cite{markdown-history}. Dále existuje řada variant Markdownu jako Kramdown či Redcarpet, které jsou oproti originálu rozšířeny o dodatečnou funkcionalitu jako poznámky pod čarou či kombinace s \gls{css}.

  Pro svou jednoduchost se jedná o populární formát zápisu dokumentací projektů, souborů README (čti mě) pro repozitáře stránek jako GitHub a GitLab, či článků některých blogů a osobních stránek.

  V projektu je v kombinaci s šablonovým jazykem Liquid používán ke psaní článků (viz. kapitola \ref{sec:Tvorba článků}).


  \subsection{\TeX} \label{sec:TeX}
  \TeX{} je programovací jazyk pro sázení odborné literatury vytvořený Donaldem E. Knuthem v 70. létech 20. století. Počáteční impulz pro vznik byla Knuthova nespokojenost s tiskovou kvalitou při vydávání jedné z jeho knih, což jej vedlo ke studiu principů sázení, tvorby fontů a později až k vytvoření vlastního sázecího systému \TeX{} \cite{tex-history}.

  Od klasických \gls{wysiwyg} programů pro tvorbu dokumentů jako MS Word či LibreOffice se liší tím, že autor při psaní nevidí, jak dokument vypadá. Namísto toho pomocí „maker” (souborů instrukcí) definuje, jak by dokument vypadat měl a \TeX{} se o samotné sázení postárá. Tento přístup klade větší důraz na obsah~--~autor se formátováním do značné míry nemusí zabývat a může se soustředit na psaní. Další výhoda \TeX u je fakt, že celý dokument je prostý text, proto jej lze verzovat.

  V projektu je používán \LaTeX{} (viz. kapitola \ref{sec:Převod webové stránky do PDF}). Jedná se o nadstavbu \TeX u, která díky pokročilejším makrům jako automatické číslování stránek, vkládání referencí, aj. umožňuje pohodlnější tvorbu dokumentů \cite{getting-started-with-latex}. K převodu \LaTeX u do formátu \gls{pdf} je používán český program pdf\TeX{} \cite{pdftex}.


  \section{Použité nástroje} \label{sec:Použité nástroje}
  Tato kapitola rozebírá v projektu použité nástroje, jejich účel a porovnání s alternativami.

  Hlavní kritérium výběru byla multiplatformnost, jelikož jsou při vývoji aktivně využívaný oba operační systémy Windows a Linux.

  % TODO: add a definition of a programming language and some basic overview


  \subsection{Git} \label{sec:Git}
  Git je otevřený\footnote{Zdrojový kód Gitu je dostupný na adrese \url{https://github.com/git/git}.} verzovací systém vytvořený Linusem Torvaldsem pro použití na operačním systému Linux. V open-source komunitě se v současné době (březen 2018) jedná o nejpoužívanější vezrovací systém \cite{version-control-usage-statistics}, což je jeden z hlavních důvodů pro jeho využití v tomto projektu.

  Verzovací systém umožňuje zaznamenávat změny a jejich důvod/význam na verzované skupině souborů a složek (tzv. repozitáři) a pomáhá tím zamezit ztrátě práce, zjednodušit hledání chyb v kódu a usnadnit spolupráci mezi autory.

  Narozdíl od centralizovaných verzovacích systémů jako Subversion či \gls{cvs} je Git decentralizovaný~--~místo jedné centrální úschovny kódu má každý svou lokální kopii s úplnou historií projektu. Výhody tohoto přístupu oproti centrální úschovně jsou např. rychlost práce s repozitářem, možnost pracovat na projektu bez internetového připojení a zamezení ztráty kódu při selhání centrálního serveru \cite{cvcs-vs-dvcs}.

  To, že má každý svou lokální kopii však nezamezuje existenci hlavní kopie. Tento fakt dokazují služby jako GitHub, GitLab či GitBucket, které tuto centrální úschovnu poskytují. % Tento fakt ... frázovat jinak


  \subsection{Jekyll} \label{sec:Jekyll}
  Jekyll je otevřený\footnote{Zdrojový kód Inkscapu je dostupný na adrese \url{https://gitlab.com/inkscape/inkscape}.} generátor statických webových stránek napsaný v jazyce Ruby. Od tradičních přístupů ke stavbě webové stránky využívajících databáze či systém pro správu obsahu se liší tím, že stránku generuje pouze z textových souborů udávajících obsah a vzhled stránky. Výsledný produkt je plně statická stránka, která není napojena na žádný dynamický systém.

  Tento přístup výrazně zjednodušuje verzování, jelikož je převážná většina projektu prostý text. Další výhoda je ochrana před potenciálními útoky, jelikož u statických webových stránek není řada frekventovaně využívaných útoků možná~--~hosting si neukládá nic, co by potenciální útočník mohl zneužít.

  Nejpopulárnější alternativna Jekyllu je Hugo, který byl při vývoji stránky rovněž testován, avšak nakonec byl kvůli aktivnější komunitě a osobní preferenci zvolen Jekyll.


  \subsection{Inkscape} \label{sec:Inkscape}
  Inkscape je otevřený\footnote{Zdrojový kód Jekyllu je dostupný na adrese \url{https://github.com/jekyll/jekyll}.} multiplaformní software na tvorbu vektorové grafiky. Oproti rastrovým softwarům, které pracují s  pixely obrázku, si vektorové pamatují informace o tom, z jakých tvarů se obrázek skládá, což z nich dělá ideální kandidáty pro tvorbu ilustrací (viz. kapitola \ref{sec:Ilustrace}).

  Jedná se o populární volbu pro ty, kteří nepoužívají profesionální software jako Adobe Illustrator kvůli vysoké ceně, uzavřenosti zdrojového kódu či nedostupnosti pro operační systém Linux.


  \subsection{Fusion 360} \label{sec:Fusion 360}
  V rámci tvorby ilustrací k odborné literatuře je kromě přístupů rozebíraných v kapitole \ref{sec:Inkscape} dále používán \gls{cad} software, který slouží ke tvorbě technických nákresů. Na rozdíl od tradičních programů na tvorbu grafiky jsou rozměry tvořených objektů přesně definovány v jednotkách \gls{si}, aby odpovídaly objektům reálného světa.

  Populární \gls{cad} software využívaný členy \gls{frc} teamů je Fusion 360 od společnosti Autodesk, převážně kvůli dostupnosti licence (zdarma pro studenty a jejich mentory). Profesionální alternativou je program SolidWorks, který je často používán studenty technických škol.

  Fusion 360 je jediný v projeku používaný program, který nefunguje na operačním systému Linux,  jelikož za něho neexistuje kvalitní multiplatformní náhrada.


  \section{Vývoj webové stránky}
  Po důkladném zvážení všech výhod a nevýhod přístupů k tvorbě vzdelávacího materiálu omezeného požadavky danými v kapitole \ref{sec:Požadavky na výsledný produkt} se webová stránka jeví jako nejvhodnější výsledný produkt. Sama o sobě totiž může při pečlivém výběru technologií splňovat jak multiplatformnost (i dostupnost ve formátu \gls{pdf}), tak otevřenost zdrojového kódu.

  Pro tvorbu stránky v rámci splnění otevřenosti zdrojového kódu byl tedy vybrán generátor statických webových stránek Jekyll (viz. kapitola \ref{sec:Jekyll}), díky čemuž lze kód produktu verzovat a volně sdílet jednodušeji, než kód webové stránky s databází.


  \subsection{Konfigurace Jekyllu}
  Po zvážení možností zdarma dostupných šablon pro Jekyll byla zvolena šablona Just the Docs (\url{https://github.com/pmarsceill/just-the-docs}), která díky své optimalizaci pro mobilní zařízení pomáhá splnit požadavek multiplatformnosti produktu. Další atraktivní atributy jsou čistý vzhled, podpora vyhledávání a přizpůsobitelné zvýrazňování syntaxe kódu.


  \subsection{Webhosting a doména} \label{sec:Webhosting a doména}
  Na českém trhu operuje řada firem, které zajišťují jak registraci domény, tak hostování webové stránky. Každá má své výhody a nevýhody, proto je volba provozovatele vysoce individuální a záleží na řadě faktorů.

  Stránka je hostována a doména zprostředkována společností WEDOS Internet a její službou NoLimit (\url{https://hosting.wedos.com/cs/webhosting.html}), zejména díky mým pozitivním zkušenostem s tímto provozovatelem v rámci minulých projektů a faktem, že ceny a parametry služeb jiných hostingů se od využívané příliš neliší.

  Jelikož je Jekyll podporován GitHubem, na kterém je hostován zdrojový kód stránky, další možností je služba GitHub Pages (\url{https://pages.github.com/}). Stránka by byla hostována zdarma, nebylo by třeba obsah nahrávat přes \gls{ftp} a generování stránky by probíhalo automaticky. Tento přístup by však znemožnil využívání Python scriptů, proto není pro účely projektu vhodný.


  \subsection{Design loga}
  Na dobré vizáži každé stránky má značný podíl její logo, které by mělo tematicky odpovídat zaměření stránky bez toho, aby bylo příliš komplikované.

  Hlavní myšlenka za vzhledem loga je spojení výrazů „RO“ a „SI“ (první dvě písmena názvu stránky) do jednoho. Toto spojení výrazy zjednodušuje a tematicky tak odpovídá obsahu webové stránky, která je založena na jednoduchosti a intuitivnosti. Tvar písmene „O“ je rovněž podobné ozubenému kolu, což do designu loga pro stránku zaměřenou na robotiku elegantně zapadá.

  \begin{figure}[H]%
    \centering

    \subfloat[Počáteční nápad]{{\includegraphics[width=.3\linewidth]{logo1.png}}}%
    \hfill
    \subfloat[Přidání ozubeného kola]{{\includegraphics[width=.3\linewidth]{logo2.png}}}%
    \hfill
    \subfloat[Konečná podoba]{{\includegraphics[width=.3\linewidth]{logo3.png}}}%

    \caption{Vývoj loga stránky}%
    \label{img:Vývoj loga stránky}%
  \end{figure}

  K designu loga byl použit program na úpravu vektorové grafiky Inkscape (viz. kapitola \ref{sec:Inkscape}).


  \section{Obsah webové stránky}
  Následující kapitola rozebírá obsah webové stránky~--~pro koho je určený, jakým způsobem probíhá jeho tvorba a jak je na stránce strukturovaný.


  \subsection{Cílové publikum}
  Jako cílové publikum jsem si vybral lidi v podobné situaci, ve které jsem byl já v roce 2017~--~jedinci se zájmem o robotiku, kteří jsou frustrovaní nedostatkem kvalitních materiálů pro studium. Materiál není kromě zájmu o robotiku omezen věkem, pohlavím či jinými osobními charakteristikami.


  \subsection{Tvorba článků} \label{sec:Tvorba článků}
  Články jsou psány o tématech, jejichž pochopení autorovi při studiu z ostatních internetových zdrojů dělalo problémy z důvodu odbornosti studijního materiálu či jeho nedostatku.

  % TODO: rephrase the first paragraph

  Po vybrání tématu článku jsou nejprve shromážděny dostupné materiály o daném tématu. Poté jsou hlavní koncepty tématu implementovány v jazyce Python a jejich funkčnost testována na robotovi ze stavebnice VEX EDR. Po úspěšném testování a odladění kódu je odvození implementace co nejjednodušším způsobem v článku vysvětleno.


  \subsubsection{Vizualizace} \label{sec:Vizualizace}
  Vizualizace jsou implementovány v jazyce JavaScript s využitím knihovny p5.js (viz. kapitola \ref{sec:p5.js}) a jsou uzpůsobeny pro ovládání myší či dotykem, aby byly funkční jak na osobních počítačích, tak na mobilech.

  Za zmínku stojí, že \gls{pdf} verze vizualizace neobsahuje, protože formát \gls{pdf} vkládání kódu kvůli bezpečnosti neumožňuje \cite{history-of-pdf}.


  \subsubsection{Ilustrace} \label{sec:Ilustrace}
  Ke tvorbě ilustrací je používán program na tvorbu vektorové grafiky Inkscape (viz. kapitola \ref{sec:Inkscape}) v kombinaci s \gls{cad} softwarem Fusion 360 (viz. kapitola \ref{sec:Fusion 360}).

  Vektorová grafika je pro vytváření ilustrací dobrá volba proto, že potřebné tvary lze vytvářet a upravovat výrazně rychleji a efektivněji než rastr. Zabírá také méně místa a je ukládána v prostém textu, což má oproti binárnímu formátu řadu výhod.

  % TODO: add illustration examples


  \subsubsection{Matematika} \label{sec:Matematika}
  Jelikož jsou probírané koncepty a algoritmy založeny na matematice, tak stránka podporuje vkládání rovnic a výrazů do článků. Tuto funkcionalitu zajišťuje JavaScriptová knihovna \KaTeX{} (viz. kapitola \ref{sec:KaTeX}), díky které lze \LaTeX ové rovnice umísťovat přímo do Markdownových článků.

  Příkladem zápisu matematiky v článku je rovnice č. \ref{eq:katex equation}, která je pomocí \KaTeX u převedena na rovnici č. \ref{eq:converted equation}.

  \begin{equation} \label{eq:katex equation}
    \verb|Je pravda, že $$\sum_{i=1}^{n} i = \frac{n(n+1)}{2}$$.|
  \end{equation}

  \begin{equation} \label{eq:converted equation}
    \text{Je pravda, že }\sum_{i=1}^{n} i = \frac{n(n+1)}{2}\text{.}
  \end{equation}


  \subsection{Navigace}
  Stránka je pro ulehčení převodu do \gls{pdf} a plynulosti navigace strukturována do značné míry jako kniha.

  Návštěvníka po zobrazení stránky přivítá \emph{úvodní stránka} (obr. \ref{img:Úvodní stránka}), která představí projekt a jeho cíle, případně obsahuje důležitá sdělení o provozu stránky. Poté následuje \emph{předmluva} (obr. \ref{img:Předmluva}), kde jsou kromě diskuzi o předpokladech pro čtení zodpovězeny otázky, které by čtenáři při čtení článků stránky mohli mít.

  \begin{figure}[H]
    \minipage{0.47\textwidth}
      \fbox{\includegraphics[width=\linewidth]{1.png}}
      \caption{Úvodní stránka} \label{img:Úvodní stránka}
    \endminipage\hfill
     \minipage{0.47\textwidth}
      \fbox{\includegraphics[width=\linewidth]{2.png}}
      \caption{Předmluva} \label{img:Předmluva}
    \endminipage
  \end{figure}

  Hlavní náplní projektu jsou \emph{tématické okruhy} (obr. \ref{img:Příklad článku 1}, \ref{img:Příklad článku 2}, \ref{img:Příklad článku 3}), které se skládají z článků probírajících koncepty s podobnou tematikou, v pořadí, ve kterém by pro návaznost měly být čteny.

  \begin{figure}[H]
    \minipage{0.3\textwidth}
      \fbox{\includegraphics[width=\linewidth]{3-1.png}}
      \caption{Příklad článku 1} \label{img:Příklad článku 1}
    \endminipage\hfill
     \minipage{0.3\textwidth}
      \fbox{\includegraphics[width=\linewidth]{3-2.png}}
      \caption{Příklad článku 2} \label{img:Příklad článku 2}
    \endminipage\hfill
     \minipage{0.3\textwidth}
      \fbox{\includegraphics[width=\linewidth]{3-3.png}}
      \caption{Příklad článku 3} \label{img:Příklad článku 3}
    \endminipage
  \end{figure}

  Na závěr je přiložen článek s \emph{odkazy na materiály} (obr. \ref{img:Odkazy na materiály}) použité při tvorbě projektu a sekce \emph{O nás} (obr. \ref{img:O nás}), která popisuje důvod za vznikem projektu, kontakty a návod na případnou spolupráci na projektu.

  \begin{figure}[H]
    \minipage{0.47\textwidth}
      \fbox{\includegraphics[width=\linewidth]{4.png}}
      \caption{Odkazy na materiály} \label{img:Odkazy na materiály}
    \endminipage\hfill
     \minipage{0.47\textwidth}
      \fbox{\includegraphics[width=\linewidth]{5.png}}
      \caption{O nás} \label{img:O nás}
    \endminipage
  \end{figure}


  \section{Funkčnost webové stránky}
  Následující kapitola pokrývá některé ze zajímavějších technických řešení použitých při tvorbě projektu.


  \subsection{Automatizace provozu} \label{sec:Automatizace provozu}
  Ke tvorbě kvalitního vzdělávacího materiálu je potřeba plynulý a do nejvyšší možné míry automatizovaný provoz stránky~--~k přidání článků by mělo stačit v příslušné složce projektu vytvořit nový soubor a zbytek procesu by měl proběhnout bez zásahu autora.

  Stránka je v tomto duchu automatizována scripty psanými v jazyce Python (viz. kapitola \ref{sec:Python}), aby se autor mohl plně soustředit na práci.


  \subsubsection{Nahrání obsahu přes FTP}
  \gls{ftp} je protokol zprostředkovávající přenos souborů mezi počítači po síti. Jedná se o populární volbu protokolu pro hostingy webových stránek a hosting tohoto projektu není výjimkou.

  Script \texttt{upload.py} se po zadání hesla připojí přes protokol \gls{ftp} na server, rekurzivně smaže soubory a adresáře aktuální verze stránky a nahraje verzi novou. Pro dodatečné zabezpečení je \gls{ip} serveru šifrována symetrickou šifrou \gls{aes}.


  \subsubsection{Generování souboru sitemap.xml}
  Soubor protokolu sitemap (\url{https://www.sitemaps.org/protocol.html}) obsahuje informace o pořadí procházení, časech změny a relativní prioritě částí stránky. Slouží vyhledávacím portálům, které tyto soubory využívají pro inteligentnější indexování stránky ve svém systému.

  Script \texttt{sitemap.py} podle pořadí článků generuje záznamy do souboru \texttt{sitemap.xml}. Informace o umístění a poslední úpravě jsou získávány z atributů souboru a priorita je přidělena podle pozice na stránce~--~úvodní stránka má prioritu $1.0$, hlavní články mají prioritu $0.8$ a vedlejší články prioritu $0.6$.


  \subsubsection{Převod webové stránky do PDF} \label{sec:Převod webové stránky do PDF}
  Pro offline dostupnost existuje mnoho různých typů souborů, na které by stránka šla převést. Často používané formáty jako \texttt{doc} a \texttt{docx} jsou však pro použití v projektu nevhodné, protože se mohou na různých zařízeních zobrazit jinak. Je tedy vhodnější použít formáty PostScript či \gls{pdf}, jejichž vizáž na prostředí závislá není \cite{history-of-pdf}.

  Přímočará varianta by tedy byla převést články do formátu \gls{pdf} programem pro převod textových formátů (jako Pandoc). Tímto přístupem by však bylo obtížné generovat obsah a úvodní stránku, a prakticky nemožné ovlivnit výsledné formátování.

  Pro potřeby projektu byl tedy zvolen převod článků do formátu \LaTeX{} (viz. kapitola \ref{sec:TeX}) a až poté do formátu \gls{pdf}, což elegantně řeší všechny nevýhody Pandocu.

  Script \texttt{tex.py} získá pořadí článků, spojí je za sebe do jednoho dokumentu a poté na ně aplikuje řadu regulárních výrazů, které provedou konverzi z formátu Markdown do formátu \LaTeX.


  \subsubsection{Komprimace obrázků}
  O komprimaci obrázků pro zmenšení velikosti stránky se stará script \texttt{compress.py}, který pomocí služby TinyPNG (\url{https://tinypng.com/}) a jejího Python \gls{api} zkomprimuje všechny obrázky vygenerované stránky. Samotný \gls{api} klíč je pro zamezení zneužití šifrován symetrickou šifrou \gls{aes}.

  Úprava fotek funguje na principu \emph{kvantování barev}~--~podobné barvy jsou spojeny do jedné, čímž lze z tradičních 24-bitových palet \gls{png} obrázků udělat palety 8-bitové a zmenšit tím obrázek bez výrazného zhoršení kvality.


  \subsubsection{Minimalizace zdrojového kódu}
  Zkompaktnění zdrojového kódu je další možná optimalizace, kterou lze načítání stránky urychlit. Prohlížečům na vzhledu kódu nezáleží a běžné uživatele nezajímá, proto jej lze na úkor čitelnosti zmenšit.

  Tuto funkci zastává script \texttt{minify.py}, který soubory typu \gls{css} a \gls{html} zmenší odstraněním komentářů, nepotřebných tagů a uvozovek, přebytečných mezer a úpravou dalších věcí, které funkcionalitu kódu nezmění.


  \subsubsection{Automatizace procesu}
  Po přidání či úpravě článku je potřeba verzi stránky na hostingu aktualizovat. Tento proces řídí script \texttt{deploy.py}, který stránku nejprve pomocí Jekyllu vygeneruje, poté ve správném pořadí spustí všechny scripty pro generování dodatečného obsahu a optimalizace, a nakonec i script pro nahrání stránky na webhosting.


  \subsection{Verzování zdrojového kódu}
  Stránka je verzována pomocí verzovacího systému Git (viz. kapitola \ref{sec:Git}). Kód je otevřený a volně dostupný přes GitHub na adrese \url{https://github.com/xiaoxiae/Robotics-Simplified-Website}.


  \subsection{Testování výkonu}

  \subsubsection{Rychlost načítání}
  Rychlost načítání webové stránky je jedním z hlavních faktorů ovlivňujících její úspěšnost. Přebytečný čas nad optimální dobu (mezi $1,8$ a $2,7$ vteřinami) má negativní vliv na procento konverzí stránky. Při zpoždění o $100$ milisekund nad tuto dobu klesá procento konverzí uživatelů osobních počítačů o $7,8 \%$. Konverze dále klesají až o $26,2 \%$ při zpoždění v řádu $2$ vteřin \cite{conversion-rate-statistics}.

  Testování bylo provedeno službou \url{https://www.webpagetest.org/}, která analyzuje webové stránky z různých lokací světa (a různých prohlížečů) a poskytuje statistiky o rychlosti načítání, množství načtených dat a mnoho dalšího.

  Obrázek \ref{img:Mapa rychlostí načítání} (vytvořený pomocí Google GeoChart \gls{api}) obsahuje informace o rychlosti načítání stránky \url{http://robotics-simplified.com/} z většiny dostupných testovacích lokalit výše uvedené služby. % TODO: přidat odkazy na testovaná data do přílohy

  \begin{figure}[H]
    \includegraphics[width=\linewidth]{map.png}
    \caption{Mapa rychlostí načítání} \label{img:Mapa rychlostí načítání}
  \end{figure}

  % TODO: add discussion about the results


  \subsubsection{Multiplatformnost}
  % TODO: how does the website work on different devices and browsers

  \section{Analytika a propagace webové stránky}
  % TODO: how is traffic analysis and marketting done

  \subsection{Analýza návštěvnosti}

  \subsubsection{Google Analytics}
  Google Analytics je služba od společnosti Google určena pro analýzu návštěvnosti webových stránek. Pro použití stačí na danou stránku přidat JavaScriptovou knihovnu

  % TODO: add location pie chart
  % TODO: add device pie chart
  % TODO: add age pie chart
  % TODO: add gender pie chart


  \subsubsection{Google Search Console}
  % TODO: what is it good foor

  \subsection{Marketting}

  \subsubsection{Sociální média}
  % TODO: add posts on ChiefDelphi, Reddit and Hacker News

  \subsubsection{FRC a FLL teamy}

  \section{Závěr}

  \newpage

  % create list of acronyms
  \glsaddall
  \glssetwidest{WYSIWYGGG}
  \setglossarystyle{alttree}
  \renewcommand{\glossarypreamble}{\vspace*{-2\baselineskip}}

  \printglossary[type=\acronymtype,title=Seznam zkratek]

  \newpage

  % create the bibliography
  \renewcommand{\refname}{Použitá literatura}
  \bibliographystyle{czechiso}
  \bibliography{soc}

  \newpage

  % create the list of figures
  \renewcommand{\listfigurename}{Seznam obrázků a tabulek}
  {%
  \let\oldnumberline\numberline%
  \renewcommand{\numberline}{\figurename~\oldnumberline}%
  \listoffigures%
  }

\end{document}
