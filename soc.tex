\documentclass[a4paper, 12pt]{article}
\usepackage[top=2.5cm, left=3.5cm, right=1.5cm, bottom=2.5cm, includefoot]{geometry}

% misc includes
\usepackage{titlesec}
\usepackage{tabularx}
\usepackage[hyphens,spaces,obeyspaces]{url}

% to automatically add date of compilation
\usepackage[ddmmyyyy]{datetime}
\renewcommand{\dateseparator}{. }

% font and encoding stuff
\usepackage[utf8]{inputenc}
\usepackage[T1]{fontenc}
\usepackage{type1ec}

% paragraph spacing and indentation
\setlength\parindent{0pt}
\setlength{\parskip}{1.2em}

% title formatting
\titleformat{\section}{\bfseries\scshape\fontsize{18}{21.6}\selectfont}{\thesection}{1em}{}
\titleformat{\subsection}{\bfseries\fontsize{16}{19.2}\selectfont}{\thesubsection}{1em}{}
\titleformat{\subsubsection}{\bfseries\fontsize{14}{16.8}\selectfont}{\thesubsubsection}{1em}{}

% paragraph and title spacing
\titlespacing{\section}{0pt}{0pt}{0.3em}
\titlespacing{\subsection}{0pt}{0pt}{0.2em}

% page numbering only starts at TOC page
\pagenumbering{gobble}

\begin{document}
  % first page's geometry is different
  \newgeometry{top=2.5cm, left=2.5cm, right=2.5cm, bottom=2.5cm}

  \bfseries

  \begin{center}
    {\fontsize{18}{21.6} \selectfont STŘEDOŠKOLSKÁ ODBORNÁ ČINNOST}\\%
    \vspace*{\baselineskip}
    {\fontsize{14}{16.8} \selectfont Obor č. 18: Informatika}\\%

    \topskip0pt
    \vspace{16em}
    {\fontsize{20}{24} \selectfont Robotika Jednoduše}%
    \vspace*{\fill}
  \end{center}

  \fontsize{16}{19.2} \selectfont
  Tomáš Sláma\\
  Liberecký Kraj
  \hfill
  Turnov 2019

  % the geometry for the rest of the document
  \newgeometry{top=2.5cm, left=3.5cm, right=1.5cm, bottom=2.5cm}

  \newpage
  \begin{center}
    {\fontsize{18}{21.6} \selectfont STŘEDOŠKOLSKÁ ODBORNÁ ČINNOST}\\%
    \vspace*{\baselineskip}
{\fontsize{14}{16.8} \selectfont Obor č. 18: Informatika}\\%

    \topskip0pt
    \vspace{10em}
    \fontsize{20}{24} \selectfont
    Robotika Jednoduše%

    Robotics Simplified%
    \vspace*{\fill}
  \end{center}

  \normalfont
  \fontsize{16}{19.6} \selectfont

  \textbf{Autoři:} Tomáš Sláma\\
  \textbf{Škola:} Gymnázium, Turnov, Jana Palacha 804, příspěvková \\ organizace, 511 21 Turnov \\
  \textbf{Kraj:} Liberecký kraj \\
  \textbf{Konzultant:} Mgr. Tomáš Novotný

  \vspace{\baselineskip}

  \fontsize{12}{14.4} \selectfont
  Turnov 2019

  \vspace{4em}

  \newpage

  \section*{\normalfont\textbf{Prohlášení}}
  Prohlašuji, že jsem svou práci SOČ vypracoval/a samostatně a použil/a jsem pouze prameny a literaturu uvedené v seznamu bibliografických záznamů.

  Prohlašuji, že tištěná verze a elektronická verze soutěžní práce SOČ jsou shodné.

  Nemám závažný důvod proti zpřístupňování této práce v souladu se zákonem č. 121/2000 Sb., o právu autorském, o právech souvisejících s právem autorským a o změně některých zákonů (autorský zákon) ve znění pozdějších předpisů.

  \qquad

  V Turnově dne \today \, ………………………….\\%
  \makebox[\linewidth]{Tomáš Sláma}%

  \newpage

  \section*{\normalfont\textbf{Poděkování}}

  \newpage

  \section*{\normalfont\textbf{Anotace}}
  Tato práce popisuje proces tvorby webové stránky \url{robotics-simplified.com}. Stránka zpracovává témata z oboru robotiky formou pochopitelnou i naprostým začátečníkem. Je poháněna generátorem statických stránek Jekyll, který doplňují JavaSriptové knihovny pro vykreslování matematických rovnic, analýzy návštěvnosti a interaktivní vizualizaci probíraných konceptů. Komprese obrázků, nahrání obsahu na hosting přes FTP, převod stránky do její knižní podoby pro offline čtení a mnoho dalšího jsou plně automatizovány pomocí scriptů psaných v jazyce Python.

  \section*{\normalfont\textbf{Klíčová slova}}
  robotika; webová stránka; vzdělávání

  \section*{\normalfont\textbf{Annotation}}
  This paper decribes the process of creating the \url{robotics-simplified.com} website. The site explores topics in the field of robotics in a beginner-friendly way. It is powered by the static site generator Jekyll, and is suplemented by JavaScript libraries used for rendering mathematical equations, analyzing the website traffic and visualizing the discussed concepts. Image compression, content upload via FTP and conversion of the website to a book version are fully automated using Python scripts. 

  \section*{\normalfont\textbf{Keywords}}
  robotics; website; education

  \newpage

  % start page numbering at the first TOC page
  \newcounter{savepage}
  \setcounter{savepage}{\value{page}}%
  \pagenumbering{arabic}
  \setcounter{page}{\numexpr\value{savepage}-1}%

  % genereate TOC (with a different name for contents)
  \renewcommand{\contentsname}{Obsah}
  \tableofcontents

  \newpage

  \section{Úvod}
  V dnešním světě plným technologií se robotika rozvíjí stále rapidnějším tempem. Roboti jsou díky brilantním programátorům a inženýrům rychlejší, obratnější a chytřejší než kdy jindy a uplatnění nalézají v nesčetném množství oborů. Jen za rok 2018 se prodej industriálních robotů zvýšil o 30 \%\cite{industrial-robot-growth} a nadnárodní korporace jako Amazon jich používají na úkor lidské pracovní síly stále více\cite{amazon-hiring}. Bez talentovaných lidí by však takový pokrok nebyl možný, proto vzniká řada programů a organizací, které se vzdělávání budoucí generace snaží podporovat.

  Značnou zásluhu má například neziskový program FIRST\cite{first-inspires} a jeho soutěže FRC (First Robotics Competition), FTC (First Technical Challenge) a FLL (First Lego League), které z robotiky dělají hru. Teamy tvořeny studenty základních a středních škol mají za úkol během několika týdnů robota od základu navrhnout, postavit a naprogramovat tak, aby plnil úkoly každým rokem se měnící výzvy.

  Teamy začátečníků a jednotlivci se zájmem o robotiku se ovšem musí potýkat s problematikou hledání studijních materiálů, ať už se jedná o programování robota či o jeho stavbu. Čtení odborných prací a luštění komplexních matematických vzorců není pro každého, zvláště pokud o daném oboru ví pramálo, a hledání kvalitních zdrojů trvá dlouho a často končí neúspěchem.

  Stránka \url{robotics-simplified.com} se snaží tento problém řešit tím, že poskytuje centralizovaný zdroj informací pro nováčky do oboru robotiky. Je strukturována jako série článků, které koncepty intuitivním způsobem vysvětlují, pro snazší pochopení jsou články obohaceny o ilustrace a interaktivní vizualizace, a jsou pro zájemce o programování doplněny zdrojovými kódy, které koncepty implementují. Celá stránka je optimalizovaná i pro mobilní zařízení a je dostupná také v knižní podobě (PDF), proto připojení k internetu není pro čtení stránky třeba.

  Tato práce se zabývá procesem tvorby, provozem a fungováním výše zmíněné webové stránky. Popisuje nástroje a programovací jazyky, které stránka využívá a opodstatňuje jejich použití oproti alternativám. Ke konci rozebírá statistiky návštěvnosti a jejich dělené podle kritérií jako věk, pohlaví, délka návštěvy stránky a jiné.

  \newpage

  \section{Požadavky na webovou stránku}
  Co by měl produkt splňovat.

  \subsection{Otevřenost zdrojového kódu}
  Možná kolaborace, sdílení zdrojového kódu...

  \subsection{Offline přístup}
  Jako kniha - čtení kdekoliv, na jakémkoliv zařízení.

  \subsection{Multiplatformnost}
  Všechny systémy, všechny zařízení.
  https://www.atrium.co/blog/founders-should-build-website-not-mobile-app/

  \subsection{Statika obsahu}
  Neřeší se zabezpečení, backend.
  Vše je text - žádné netrackovatelné binární části.

  \section{Použité programovací jazyky}
  Jaké programovací jazyky jsem použil při stavění projektů.
  Popis jejich funkce, jejich využití v projektu a jejich možné náhrady.

  \subsection{JavaScript}

  \subsubsection{p5.js}

  \subsubsection{MathJax}

  \subsection{CSS}
  Sass.

  \subsection{HTML}

  \subsection{Python}

  \subsection{\LaTeX{}}

  \subsection{Markdown}


  \section{Použité nástroje}
  Jaké nástroje jsem použil při stavění projektů.
  Popis jejich funkce, jejich využití v projektu a jejich možné náhrady.

  \subsection{Jekyll}

  \subsection{Atom}

  \subsection{Inkscape, GIMP}

  \subsection{Fusion 360}

  \subsection{Git, GitKraken}

  \section{Vývoj webové stránky}

  \subsection{Konfigurace Jekyllu}
  Hledání themu.
  Konfigurace themu.

  \subsection{Webhosting a doména}
  Wedos a jejich NoLimit service.
  Alternativy - git hostování.

  \subsection{Design}

  \subsubsection{Tvorba loga}
  Myšlenka za logem, iterace...

  \subsubsection{Zvýraznění syntaxe kódu}

  \section{Obsah webové stránky}

  \subsection{Cílové publikum}

  \subsection{Tvorba článků}
  Jak probíhá psaní článků.
  Jak vypadá struktura článku - přirovnání k Wikipedii.

  \subsection{Struktura obsahu}

  \section{Funkcionalita webové stránky}

  \subsection{Automatizace}
  Napsat o deploy.py, který celý proces automatizuje.

  \subsubsection{Nahrání obsahu přes FTP}
  FTP připojení, vše smaže a poté to tam rekurzivně zase nahraje.

  \subsubsection{Generování souboru sitemap.xml}

  \subsubsection{Převod webové stránky do PDF}
  Generování přes regexy.

  \subsubsection{Komprimace obrázků}
  http://www.tinypdf.com/

  \subsection{Verzování}
  Jak probíhá verzování a přes jaký verzovací systém je prováděno.
  Kam je

  \section{Provoz webové stránky}
  Jak probíhá provoz serveru.

  \subsection{Analýza návštěvnosti}

  \subsubsection{Google Analytics}
  Na co to je
  Proč Analytics
  Statistiky, grafy

  \subsubsection{Google Search Console}
  Na co to je
  Proč Search
  Statistiky, grafy

  \subsection{Marketting}

  \subsubsection{Sociální média}
  Reddit, Hacker News, Facebook...

  \subsubsection{FRC a FLL teamy}

  \subsection{Testování}
  https://www.webpagetest.org/

  \subsubsection{Rychlost načítání}
  Kolik dat stránka načítá.
  Jak rychle se načítá z různých zemí světa.

  \subsubsection{Multiplatformnost}
  Jak to funguje na různých prohlížečích.
  Jak to funguje na různých platformách.

  \section{Závěr}

\end{document}
