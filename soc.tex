\documentclass[a4paper, 12pt]{article}
\usepackage[top=2.5cm, left=3.5cm, right=1.5cm, bottom=2.5cm, includefoot]{geometry}

% image and caption include setup
\usepackage{graphicx}
\graphicspath{ {./img/} }

% misc includes
\usepackage{titlesec}
\usepackage{tabularx}

% link macros
\usepackage[hyphens,spaces,obeyspaces]{url}
\usepackage[hidelinks,unicode]{hyperref}

% for numbering bibliography, list of figures and for removing TOC reference from TOC
\usepackage[numbib, nottoc]{tocbibind}
\makeatletter
\renewcommand\listoffigures{%
    \section{\listfigurename}%
    \@mkboth{\MakeUppercase\listfigurename}%
        {\MakeUppercase\listfigurename}%
    \@starttoc{lof}%
}
\makeatother

% figure caption setup
\usepackage[font=small,labelfont=bf]{caption}
\renewcommand{\figurename}{Obr. č.}

% to automatically add date of compilation
\usepackage[dmyyyy]{datetime}
\renewcommand{\dateseparator}{. }

% font and encoding stuff
\usepackage[utf8]{inputenc}
\usepackage[T1]{fontenc}
\usepackage{type1ec}

% paragraph spacing and indentation
\setlength\parindent{0pt}
\setlength{\parskip}{1.2em}

% title formatting
\titleformat{\section}{\bfseries\scshape\fontsize{18}{21.6}\selectfont}{\thesection}{1em}{}
\titleformat{\subsection}{\bfseries\fontsize{16}{19.2}\selectfont}{\thesubsection}{1em}{}
\titleformat{\subsubsection}{\bfseries\fontsize{14}{16.8}\selectfont}{\thesubsubsection}{1em}{}

% paragraph and title spacing
\titlespacing{\section}{0pt}{0pt}{0.3em}
\titlespacing{\subsection}{0pt}{0pt}{0.2em}

% page numbering only starts at TOC page
\pagenumbering{gobble}

\begin{document}
  % first page's geometry is different
  \newgeometry{top=2.5cm, left=2.5cm, right=2.5cm, bottom=2.5cm}

  \bfseries

  \begin{center}
    {\fontsize{18}{21.6} \selectfont STŘEDOŠKOLSKÁ ODBORNÁ ČINNOST}\\%
    \vspace*{\baselineskip}
    {\fontsize{14}{16.8} \selectfont Obor č. 18: Informatika}\\%

    \topskip0pt
    \vspace{16em}
    {\fontsize{20}{24} \selectfont Robotika Jednoduše}%
    \vspace*{\fill}
  \end{center}

  \fontsize{16}{19.2} \selectfont
  Tomáš Sláma\\
  Liberecký Kraj
  \hfill
  Turnov 2019

  % the geometry for the rest of the document
  \newgeometry{top=2.5cm, left=3.5cm, right=1.5cm, bottom=2.5cm}

  \newpage
  \begin{center}
    {\fontsize{18}{21.6} \selectfont STŘEDOŠKOLSKÁ ODBORNÁ ČINNOST}\\%
    \vspace*{\baselineskip}
    {\fontsize{14}{16.8} \selectfont Obor č. 18: Informatika}\\%

    \topskip0pt
    \vspace{10em}
    \fontsize{20}{24} \selectfont
    Robotika Jednoduše%

    Robotics Simplified%
    \vspace*{\fill}
  \end{center}

  \normalfont
  \fontsize{16}{19.6} \selectfont

  \textbf{Autoři:} Tomáš Sláma\\
  \textbf{Škola:} Gymnázium, Turnov, Jana Palacha 804, příspěvková \\ organizace, 511 21 Turnov \\
  \textbf{Kraj:} Liberecký kraj \\
  \textbf{Konzultant:} Mgr. Tomáš Novotný

  \vspace{\baselineskip}

  \fontsize{12}{14.4} \selectfont
  Turnov 2019

  \vspace{4em}

  \newpage

  \section*{\normalfont\textbf{Prohlášení}}
  Prohlašuji, že jsem svou práci SOČ vypracoval/a samostatně a použil/a jsem pouze prameny a literaturu uvedené v seznamu bibliografických záznamů.

  Prohlašuji, že tištěná verze a elektronická verze soutěžní práce SOČ jsou shodné.

  Nemám závažný důvod proti zpřístupňování této práce v souladu se zákonem č. 121/2000 Sb., o právu autorském, o právech souvisejících s právem autorským a o změně některých zákonů (autorský zákon) ve znění pozdějších předpisů.

  \qquad

  V Turnově dne \today \, ………………………….\\%
  \makebox[\linewidth]{Tomáš Sláma}%

  \newpage

  \section*{\normalfont\textbf{Poděkování}}

  \newpage

  \section*{\normalfont\textbf{Anotace}}
  Tato práce popisuje proces tvorby webové stránky \url{robotics-simplified.com}. Stránka zpracovává témata z oboru robotiky formou pochopitelnou i naprostým začátečníkem. Je poháněna generátorem statických stránek Jekyll, který doplňují JavaSriptové knihovny pro vykreslování matematických rovnic, analýzy návštěvnosti a interaktivní vizualizaci probíraných konceptů. Python scripty automatizují kompresi obrázků, generování souboru sitemap, nahrání obsahu na hosting přes FTP a převod stránky do knižní podoby pro offline čtení.

  \subsection*{\fontsize{18}{21.6}\selectfont\normalfont\textbf{Klíčová slova}}
  robotika; webová stránka; vzdělávání

  \section*{\normalfont\textbf{Annotation}}
  This paper decribes the process of creating the \url{robotics-simplified.com} website. The site covers topics in the field of robotics in a beginner-friendly way. It is powered by the static site generator Jekyll, and is suplemented by JavaScript libraries used for rendering mathematical equations, analyzing the website traffic and creating interactive visualizations of the discussed concepts. Python scripts fully automate image compression, sitemap file generation, content upload via FTP and conversion to a book version for offline reading.

  \section*{\normalfont\textbf{Keywords}}
  robotics; website; education

  \newpage

  % start page numbering at the first TOC page
  \newcounter{savepage}
  \setcounter{savepage}{\value{page}}%
  \pagenumbering{arabic}
  \setcounter{page}{\numexpr\value{savepage}-1}%

  % genereate TOC (with a different name for contents)
  \renewcommand{\contentsname}{Obsah}
  \tableofcontents

  \newpage

  \section{Úvod}
  V dnešním světě plném technologií se robotika rozvíjí stále rapidnějším tempem. Roboti jsou díky brilantním programátorům a inženýrům rychlejší, obratnější a chytřejší než kdy jindy a uplatnění nalézají v nesčetném množství oborů. Jen za rok 2018 se prodej industriálních robotů zvýšil o 30 \%\cite{industrial-robot-growth} a nadnárodní korporace jako Amazon jich používají na úkor lidské pracovní síly stále více\cite{amazon-hiring}. Bez talentovaných lidí by však takový pokrok nebyl možný, proto vzniká řada programů a organizací, které se vzdělávání budoucí generace snaží podporovat.

  Značnou zásluhu má například neziskový program FIRST\cite{first-inspires} a jeho soutěže FRC (First Robotics Competition), FTC (First Technical Challenge) a FLL (First Lego League), které z robotiky dělají hru. Týmy tvořeny studenty základních a středních škol mají za úkol během několika týdnů robota od základu navrhnout, postavit a naprogramovat tak, aby plnil úkoly každým rokem se měnící výzvy.

  Týmy začátečníků a jednotlivci se zájmem o robotiku se však musí potýkat s problematikou hledání studijních materiálů, ať už se jedná o programování robota či o jeho stavbu. Čtení odborných prací a luštění komplexních matematických vzorců není pro každého, zvláště pokud o daném oboru ví pramálo, a hledání kvalitních zdrojů trvá dlouho a často končí neúspěchem.

  Stránka \url{robotics-simplified.com} se snaží tento problém řešit tím, že poskytuje centralizovaný zdroj informací pro nováčky do oboru robotiky. Je strukturována jako série článků, které koncepty intuitivním způsobem vysvětlují, pro snazší pochopení jsou články obohaceny o ilustrace a interaktivní vizualizace, a jsou pro zájemce o programování doplněny zdrojovými kódy, které koncepty implementují. Celá stránka je rovněž optimalizovaná pro mobilní zařízení a je dostupná také v knižní podobě (ve formátu PDF), proto připojení k internetu není pro čtení stránky třeba.

  Tato práce se zabývá procesem tvorby, provozem a fungováním výše zmíněné webové stránky. Popisuje nástroje a programovací jazyky, které stránka využívá a opodstatňuje jejich použití oproti alternativám. Ke konci rozebírá statistiky návštěvnosti a jejich dělení podle kritérií jako věk, pohlaví, délka návštěvy stránky a jiné.

  \newpage

  \section{Požadavky na výsledný produkt}
  Myšlenka tvorby kvalitních materiálů pro studium robotiky je příliš rozsáhlá na to, aby mohla práce na projektu začít bez větší úvahy: bude se jednat o webovou stránku, aplikaci, či repozitář se zdrojovými kódy?

  Před začátkem je třeba definovat požadavky, které by výsledný produkt měl splňovat. Podle těch lze způsoby provedení porovnat a použít ten, který se jeví za nejvhodnější.


  \subsection{Otevřenost zdrojového kódu}
  Hlavní požadavek, který by produkt měl splňovat je otevřenost (bezplatná dostupnost) zdrojového kódu. Otevřenost výrazně usnadňuje kolaboraci (kdokoliv může navrhnout změnu) a zvyšuje důvěru v daný software (chování programu lze snadno ověřit nahlednutím do kódu).

  Příklady populárního otevřeného softwaru využívaného při stavbě tohoto projektu jsou např. Inkscape\cite{inkscape-source}, Atom\cite{atom-source}, Git\cite{git-source} nebo Jekyll\cite{jekyll-source}, které jsou podrobněji rozebírány v kapitole \ref{sec:Použité nástroje}.


  \subsection{Multiplatformnost}
  Co se dostupnosti týče, důležitým požadavkem je multiplatformnost, tj. možnost přístupu k produktu z více než jedné platformy\footnote{Slovem platforma je tomto kontextu myšlen způsob přístup na stránku jako tablet, stolní počítač, mobil, čtečka elektronických knih, aj.}. Výsledný produkt musí být multiplatformní primárně kvůli mému osobnímu názoru, že vzdělání by nemělo být omezeno technologickými preferencemi jednotlivce.

  Tato podmínka fakticky znemožňuje tvorbu aplikace, jelikož by se jak finanční\cite{apple-store-membership,android-store-membership,wedos-hosting}, tak časové nároky mnohonásobně zvýšily a komplexita projektu by přerostla nad rámec jednotlivce.


  \section{Použité programovací jazyky}
  Jaké programovací jazyky jsem použil při stavění projektů.
  Popis jejich funkce, jejich využití v projektu a jejich možné náhrady.

  \subsection{JavaScript}

  \subsubsection{p5.js}

  \subsubsection{MathJax}

  \subsection{CSS}
  Sass.

  \subsection{HTML}

  \subsection{Python}

  \subsection{Markdown} \label{sec:Markdown}

  \subsection{Liquid}

  \subsection{\TeX{}} \label{sec:TeX}
  \TeX{} je programovací jazyk pro sázení odborné literatury vytvořený Donaldem E. Knuthem v 70. létech 20. století. Počáteční impulz pro vznik byla Knuthova nespokojenost s tiskovou kvalitou při vydávání jedné z jeho knih, což jej vedlo ke studiu principů sázení, tvorby fontů a později až k vytvoření vlastního sázecího systému \TeX\cite{tex-history}.

  Od klasických WYSIWYG (what you see is what you get) programů pro tvorbu dokumentů jako MS Word či LibreOffice se liší tím, že autor při psaní nevidí, jak dokument vypadá. Namísto toho pomocí „maker” (souborů instrukcí) definuje, jak by dokument měl vypadat a \TeX{} se o samotné sázení postárá. Tento princip klade větší důraz na obsah -- autor se formátováním do značné míry nemusí zabývat.

  V projektu je používán \LaTeX. Jedná se o nadstavbu \TeX u, která díky pokročilejším makrům (automatické číslování stránek, vkládání referencí, aj.) umožňuje pohodlnější tvorbu dokumentů \cite{getting-started-with-latex}. K převodu \LaTeX u do formátu PDF je použit český program pdf\TeX{}\cite{pdftex}.


  \section{Použité nástroje} \label{sec:Použité nástroje}
  Jaké nástroje jsem použil při stavění projektů.
  Popis jejich funkce, jejich využití v projektu a jejich možné náhrady.

  \subsection{Git} \label{sec:Git}
  Git je otevřený\cite{git-source} verzovací systém vytvořen Linusem Torvaldsem pro použití na operačním systému Linux. V open-source komunitě se v současné době jedná o nejpoužívanější software tohoto typu\cite{version-control-usage-statistics}, což je jeden z důvodů pro jeho využití v tomto projektu.

  Verzovací systém umožňuje zaznamenávat změny (a jejich důvod/význam) na vezrované skupině souborů a složek (tzv. repozitáři), aby se zamezilo ztrátě práce, zjednodušilo se hledání bugů (chyb) v kódu a usnadnila se kolaborace.

  Narozdíl od centralizovaných verzovacích systémů jako Subversion a CVS je Git decentralizovaný -- místo jedné centrální úschovny kódu má každý svou lokální kopii s úplnou historií projektu. Výhody tohoto přístupu jsou např. rychlost práce s repozitářem, možnost pracovat na projektu bez internetového připojení a zamezení ztráty kódu při selhání centrálního serveru\cite{cvcs-vs-dvcs}.

  To, že má každý svou lokální kopii však nezamezuje existenci hlavního repozitáře. Tento fakt dokazují služby jako GitHub, GitLab či GitBucket, které tuto centrální úschovnu poskytují.


  \subsection{Jekyll} \label{sec:Jekyll}
  Jekyll je otevřený\cite{jekyll-source} generátor statických webových stránek napsán v jazyce Ruby. Od tradičních přístupů ke stavbě webové stránky využívajících databáze či systém pro správu obsahu se liší tím, že stránku generuje pouze z textových souborů udávajících obsah a styl stránky. Výsledný produkt je statická stránka, která není napojena na žádný dynamický systém.

  Tento přístup výrazně zjednodušuje verzování, jelikož je převážná většina souborů projektu text. Další výhoda je ochrana před potenciálními útoky, protože u statických webových stránek není mnoho frekventovaně používaných útoků možných -- produkt si neukládá nic, co by útočník mohl zneužít.

  Nejpopulárnější alternativna je Hugo, který byl při vývoji stránky rovněž testován, avšak nakonec byl kvůli aktivnější komunitě zvolen Jekyll.


  \subsection{Inkscape} \label{sec:Inkscape}
  Většinu softwaru na tvorbu grafiky lze rozdělit do dvou kategorií v závislosti na tom, s jakým typem grafiky programy pracují\cite{vector-vs-bitmap}.

  \emph{Vektorová} grafika pracuje s útvary jako čáry a křivky, o kterých si program pamatuje jejich vlastnosti (pozice, tloušťka, barva, atd.). Rozlišení může tím pádem být libovolně veliké, neboť je každý tento abstraktní útvar možno vykreslit do potřebného detailu.

  \emph{Bitmapová} grafika si naopak pamatuje samotné pixely: jaká je barva každého bodu obrázku. Dosažené libovolné kvality tedy není možná, jelikož si program pamatuje pouze pevně dané body.

  % doplnit, proč je na ilustrace vektor lepší

  Inkscape je otevřený\cite{inkscape-source} multiplaformní software na tvorbu vektorové grafiky. Jedná se o populární volbu pro ty, kteří nepoužívají profesionální software jako Adobe Illustrator z důvodů ceny či nedostupnosti pro Linux\cite{illustrator-pricing}.


  \subsection{Fusion 360}
  V rámci tvorby ilustrací k odborné literatuře je kromě tradičních přístupů rozebíraných v kapitole \ref{sec:Inkscape} používaný CAD (computer-aided design) software, který slouží ke tvorbě technických nákresů. Na rozdíl od tradičních programů na tvorbu grafiky jsou rozměry tvořených objektů přesně definovány v jednotkách SI, aby odpovídaly objektům reálného světa.

  V komunitě soutěže FRC je populární Fusion 360 od společnosti Autodesk, převážně protože je licence pro studenty (a jejich mentory) zdarma \cite{fusion-360-pricing}. Profesionální alternativou je program SolidWorks, který je většinou poskytován školami kvůli jeho ceně pro jednotlivce\cite{solidworks-pricing}.

  Stojí za zmínku, že Fusion 360 je jediný v projektu použitý program, který nefunguje na operačním systému Linux a to z toho důvodu, že neexistuje kvalitní multiplatformní náhrada.


  \section{Vývoj webové stránky}

  \subsection{Konfigurace Jekyllu}
  Hledání themu.
  Konfigurace themu.

  \subsection{Webhosting a doména} \label{sec:Webhosting a doména}
  Na českém trhu operuje velké množství firem, které zajišťují jak registraci domény, tak hostování samotné stránky. Každá má své výhody a nevýhody, proto je volba provozovatele vysoce individuální a záleží na řadě faktorů.

  Stránka je hostována a doména zprostředkována společností WEDOS Internet a její službou NoLimit\cite{wedos-hosting}, zejména kvůli mým pozitivním zkušenostem v rámci minulých projektů a faktem, že ceny a parametry jiných hostingů se od využívaného pro menší projekty příliš neliší.

  Jelikož je Jekyll podporován GitHubem, který hostuje zdrojový kód stránky, další možností je služba GitHub Pages\cite{github-pages}. Stránka by byla hostována zdarma, nebylo by třeba obsah nahrávat přes FTP a po provedení změny na projektu by generování stránky probíhalo automaticky. Tento přístup by však znemožnil využívání scriptů pro stavbu stránky, proto není pro účely projektu vhodný.


  \subsection{Design loga}
  Na dobré vizáži každé stránky má značný podíl její logo, které by mělo tematicky odpovídat zaměření stránky bez toho, aby působilo příliš složitě.

  Hlavní koncept, na kterém za je logo postaveno je spojení výrazů „RO“ a „SI“ (první dva znaky jména projektu) do jednoho, čímž dochází k tématice zjednodušení, na které je stránka založena. Tvar znaku „O“ je rovněž podobný ozubenému kolu, což do designu loga pro stránku zaměřenou na robotiku elegantně zapadá.

  \begin{center}
    \includegraphics[width=0.9\textwidth]{logo}
    \captionof{figure}{Vývoj loga stránky}
    \label{img:Vývoj loga stránky}
  \end{center}

  Obrázek č.~\ref{img:Vývoj loga stránky} znázorňuje proces tvorby loga: počáteční nápad, následné přidání ozubeného kola a drobné úpravy na závěr.

  Ke tvorbě loga byl použit program na úpravu vektorové grafiky Inkscape (viz. kapitola \ref{sec:Inkscape}).


  \section{Obsah webové stránky}

  \subsection{Cílové publikum}

  \subsection{Tvorba článků}
  Jak probíhá psaní článků.
  Jak vypadá struktura článku - přirovnání k Wikipedii.

  \subsection{Struktura obsahu}

  \section{Funkčnost webové stránky}

  \subsection{Automatizace provozu}
  Napsat o deploy.py, který celý proces automatizuje.

  \subsubsection{Nahrání obsahu přes FTP}
  FTP připojení, vše smaže a poté to tam rekurzivně zase nahraje.

  \subsubsection{Generování souboru sitemap.xml}

  \subsubsection{Převod webové stránky do PDF}
  Generování přes regexy.

  \subsubsection{Komprimace obrázků}
  http://www.tinypdf.com/

  \subsection{Verzování zdrojového kódu}
  Jak probíhá verzování a přes jaký verzovací systém je prováděno.
  Kam je

  \section{Provoz webové stránky}
  Jak probíhá provoz serveru.

  \subsection{Analýza návštěvnosti}

  \subsubsection{Google Analytics}
  Na co to je
  Proč Analytics
  Statistiky, grafy

  \subsubsection{Google Search Console}
  Na co to je
  Proč Search
  Statistiky, grafy

  \subsection{Marketting}

  \subsubsection{Sociální média}
  Reddit, Hacker News, Facebook...

  \subsubsection{FRC a FLL teamy}

  \subsection{Testování výkonu}
  https://www.webpagetest.org/

  \subsubsection{Rychlost načítání}
  Kolik dat stránka načítá.
  Jak rychle se načítá z různých zemí světa.

  \subsubsection{Multiplatformnost}
  Jak to funguje na různých prohlížečích.
  Jak to funguje na různých platformách.

  \section{Závěr}

  \newpage

  % create the bibliography
  \renewcommand{\refname}{Použitá literatura}
  \bibliographystyle{czechiso}
  \bibliography{soc}

  % create the list of figures
  \renewcommand{\listfigurename}{Seznam obrázků a tabulek}
  {%
  \let\oldnumberline\numberline%
  \renewcommand{\numberline}{\figurename~\oldnumberline}%
  \listoffigures%
  }

\end{document}
