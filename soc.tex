\documentclass[a4paper, 12pt]{article}
\usepackage[top=2.5cm, left=3.5cm, right=1.5cm, bottom=2.5cm, includefoot]{geometry}

\usepackage[czech]{babel}                     % use the czech language

\usepackage{graphicx}                         % graphics
\graphicspath{ {./img/} }                     % graphics path
\usepackage{float}                            % floats in correct position

\usepackage{titlesec}                         % title
\usepackage{tabularx}                         % tables
\usepackage[hyphens,spaces,obeyspaces]{url}   % url
\usepackage[hidelinks,unicode]{hyperref}      % interactive links
\usepackage{amsmath, amstext}                 % math

\usepackage[font=small,labelfont=bf]{caption} % caption settings
\usepackage[figurename=Obr. č.]{caption}   % caption name

\usepackage[numbib, nottoc]{tocbibind}        % number bib, no toc in toc
\makeatletter                                 % lof numbering
\renewcommand\listoffigures{%
    \section{\listfigurename}%
    \@mkboth{\MakeUppercase\listfigurename}%
        {\MakeUppercase\listfigurename}%
    \@starttoc{lof}%
}
\makeatother

\usepackage[dmyyyy]{datetime}                 % for dmyyyy datetimes
\renewcommand{\dateseparator}{. }             % change datetime separator

% encoding and font stuff
\usepackage[T1]{fontenc}
\usepackage{fontspec}
\usepackage{type1ec}

% paragraph spacing and indentation
\setlength\parindent{0pt}
\setlength{\parskip}{1.2em}

% title formatting
\titleformat{\section}{\bfseries\scshape\fontsize{18}{21.6}\selectfont}{\thesection}{1em}{}
\titleformat{\subsection}{\bfseries\fontsize{16}{19.2}\selectfont}{\thesubsection}{1em}{}
\titleformat{\subsubsection}{\bfseries\fontsize{14}{16.8}\selectfont}{\thesubsubsection}{1em}{}

% paragraph and title spacing
\titlespacing{\section}{0pt}{0pt}{0.3em}
\titlespacing{\subsection}{0pt}{0pt}{0.2em}

% prevent widow and orphan lines
\widowpenalty10000
\clubpenalty10000

% page numbering only starts at TOC page
\pagenumbering{gobble}

% define KaTeX logo
\makeatletter
\DeclareRobustCommand{\KaTeX}{%
  K\kern -.19em
  {\sbox \z@ T\vbox to\ht \z@ {\hbox{%
  \check@mathfonts
  \fontsize\sf@size\z@
  \selectfont A}%
  \vss}%
}\kern -.15em
\TeX}
\makeatother

\begin{document}
  % first page's geometry is different
  \newgeometry{top=2.5cm, left=2.5cm, right=2.5cm, bottom=2.5cm}

  \bfseries

  \begin{center}
    {\fontsize{18}{21.6} \selectfont STŘEDOŠKOLSKÁ ODBORNÁ ČINNOST}\\%
    \vspace*{\baselineskip}
    {\fontsize{14}{16.8} \selectfont Obor č. 18: Informatika}\\%

    \topskip0pt
    \vspace{16em}
    {\fontsize{20}{24} \selectfont Robotika Jednoduše}%
    \vspace*{\fill}
  \end{center}

  \fontsize{16}{19.2} \selectfont
  Tomáš Sláma\\
  Liberecký Kraj
  \hfill
  Turnov 2019

  % the geometry for the rest of the document
  \newgeometry{top=2.5cm, left=3.5cm, right=1.5cm, bottom=2.5cm}

  \newpage
  \begin{center}
    {\fontsize{18}{21.6} \selectfont STŘEDOŠKOLSKÁ ODBORNÁ ČINNOST}\\%
    \vspace*{\baselineskip}
    {\fontsize{14}{16.8} \selectfont Obor č. 18: Informatika}\\%

    \topskip0pt
    \vspace{10em}
    \fontsize{20}{24} \selectfont
    Robotika Jednoduše%

    Robotics Simplified%
    \vspace*{\fill}
  \end{center}

  \normalfont
  \fontsize{16}{19.6} \selectfont

  \textbf{Autoři:} Tomáš Sláma\\
  \textbf{Škola:} Gymnázium, Turnov, Jana Palacha 804, příspěvková \\ organizace, 511 21 Turnov \\
  \textbf{Kraj:} Liberecký kraj \\
  \textbf{Konzultant:} Mgr. Tomáš Novotný

  \vspace{\baselineskip}

  \fontsize{12}{14.4} \selectfont
  Turnov 2019

  \vspace{4em}

  \newpage

  \section*{\normalfont\textbf{Prohlášení}}
  Prohlašuji, že jsem svou práci SOČ vypracoval/a samostatně a použil/a jsem pouze prameny a literaturu uvedené v seznamu bibliografických záznamů.

  Prohlašuji, že tištěná verze a elektronická verze soutěžní práce SOČ jsou shodné.

  Nemám závažný důvod proti zpřístupňování této práce v souladu se zákonem č. 121/2000 Sb., o právu autorském, o právech souvisejících s právem autorským a o změně některých zákonů (autorský zákon) ve znění pozdějších předpisů.

  \qquad

  V Turnově dne \today \, ………………………….\\%
  \makebox[\linewidth]{Tomáš Sláma}%

  \newpage

  \section*{\normalfont\textbf{Poděkování}}
  V první řadě bych rád poděkoval své rodině a přátelům za neustálou podporu a pozitivitu při tvorbě projektu. Za konzultace o obsahu, struktuře a použitých technologiích děkuji Lukáši Pitoňákovi, Janu Poláškovi a Tomáši Novotnému. Také děkuji Kateřině Sulkové za rady ke psaní samotné SOČ práce a na závěr všem návštěvníkům projektu, kteří svou přítomností a feedbackem pomáhají v jeho rozvoji.

  \newpage

  \section*{\normalfont\textbf{Anotace}}
  Tato práce popisuje proces tvorby webové stránky \url{robotics-simplified.com}. Stránka zpracovává témata z oboru robotiky formou pochopitelnou i naprostým začátečníkem. Je poháněna generátorem statických stránek Jekyll, který doplňují JavaSriptové knihovny pro vykreslování matematických rovnic, analýzy návštěvnosti a interaktivní vizualizaci probíraných konceptů. Python scripty automatizují kompresi obrázků, generování souboru sitemap, nahrání obsahu na hosting přes FTP a převod stránky do knižní podoby pro offline čtení.

  \subsection*{\fontsize{18}{21.6}\selectfont\normalfont\textbf{Klíčová slova}}
  robotika; webová stránka; vzdělávání

  \section*{\normalfont\textbf{Annotation}}
  This paper decribes the process of creating the \url{robotics-simplified.com} website. The site covers topics in the field of robotics in a beginner-friendly way. It is powered by the static site generator Jekyll, and is suplemented by JavaScript libraries for rendering mathematical equations, analyzing the website traffic and creating interactive visualizations of the discussed concepts. Python scripts fully automate image compression, sitemap file generation, content upload via FTP and conversion to a book version for offline reading.

  \section*{\normalfont\textbf{Keywords}}
  robotics; website; education

  \newpage

  % start page numbering at the first TOC page
  \newcounter{savepage}
  \setcounter{savepage}{\value{page}}%
  \pagenumbering{arabic}
  \setcounter{page}{\numexpr\value{savepage}-1}%

  % genereate TOC (with a different name for contents)
  \renewcommand{\contentsname}{Obsah}
  \tableofcontents

  \newpage

  \section{Úvod}
  V dnešním světě plném technologií se robotika rozvíjí stále rapidnějším tempem. Roboti jsou díky brilantním programátorům a inženýrům rychlejší, obratnější a chytřejší než kdy jindy a uplatnění nalézají v nesčetném množství oborů. Jen za rok 2018 se prodej industriálních robotů zvýšil o 30 \%\cite{industrial-robot-growth} a nadnárodní korporace jako Amazon jich používají na úkor lidské pracovní síly stále více\cite{amazon-hiring}. Bez talentovaných lidí by však takový pokrok nebyl možný, proto vzniká řada programů a organizací, které se vzdělávání budoucí generace snaží podporovat.

  Značnou zásluhu má například neziskový program FIRST (https://www.firstinspires.org/) a jeho soutěže FRC (First Robotics Competition), FTC (First Technical Challenge) a FLL (First Lego League), které z robotiky dělají hru. Týmy tvořeny studenty základních a středních škol mají za úkol během několika týdnů robota od základu navrhnout, postavit a naprogramovat tak, aby plnil úkoly každým rokem se měnící výzvy.

  Týmy začátečníků a jednotlivci se zájmem o robotiku se však musí potýkat s problematikou hledání studijních materiálů, ať už se jedná o programování robota či o jeho stavbu. Čtení odborných prací a luštění komplexních matematických vzorců není pro každého, zvláště pokud o daném oboru ví pramálo, a hledání kvalitních zdrojů trvá dlouho a často končí neúspěchem.

  Stránka \url{robotics-simplified.com} se snaží tento problém řešit tím, že poskytuje centralizovaný zdroj informací pro nováčky do oboru robotiky. Je strukturována jako série článků, které koncepty intuitivním způsobem vysvětlují, pro snazší pochopení jsou články obohaceny o ilustrace a interaktivní vizualizace, a jsou pro zájemce o programování doplněny zdrojovými kódy, které koncepty implementují. Celá stránka je rovněž optimalizovaná pro mobilní zařízení a je dostupná také v knižní podobě (ve formátu PDF), proto připojení k internetu není pro čtení stránky třeba.

  Tato práce se zabývá procesem tvorby, provozem a fungováním výše zmíněné webové stránky. Popisuje nástroje a programovací jazyky, které stránka využívá a opodstatňuje jejich použití oproti alternativám. Ke konci rozebírá statistiky návštěvnosti a jejich dělení podle kritérií jako věk, pohlaví, délka návštěvy stránky a jiné.

  \newpage

  \section{Požadavky na výsledný produkt}
  Myšlenka tvorby kvalitních materiálů pro studium robotiky je příliš rozsáhlá na to, aby mohla práce na projektu začít bez větší úvahy: bude se jednat o webovou stránku, aplikaci, knihu, či repozitář se zdrojovými kódy?

  Před začátkem je třeba definovat požadavky, které by výsledný produkt měl splňovat. Podle těch lze způsoby provedení porovnat a použít ten, který se jeví za nejvhodnější.


  \subsection{Otevřenost zdrojového kódu}
  Hlavní požadavek, který by produkt měl splňovat je \emph{otevřenost} (bezplatná dostupnost) zdrojového kódu. Otevřenost výrazně usnadňuje kolaboraci (kdokoliv může navrhnout změnu) a zvyšuje důvěru v daný software (chování programu lze snadno ověřit nahlednutím do kódu).

  Příklady populárního otevřeného softwaru využívaného při stavbě tohoto projektu jsou např. Inkscape, Atom, Git či Jekyll, které jsou podrobněji rozebírány v kapitole \ref{sec:Použité nástroje}.


  \subsection{Multiplatformnost}
  Co se dostupnosti týče, důležitým požadavkem je \emph{multiplatformnost}, tj. možnost přístupu k produktu z více než jedné platformy jako stolní počítač, mobil, čtečka elektronických knih, aj. Výsledný produkt musí být multiplatformní primárně kvůli mému osobnímu názoru, že vzdělání by nemělo být omezeno technologickými preferencemi jednotlivce.

  Tato podmínka fakticky znemožňuje tvorbu aplikace, jelikož by finanční i časové nároky stavby aplikací pro vícero systémů přerostly nad možnosti jednotlivce.


  \section{Použité programovací jazyky}
  Následující kapitola rozebírá historii programovacích jazyků a jejich použití v projektu.


  \subsection{JavaScript}
  Okolo roku 1995 bylo pro internetový prohlížeč Netscape potřeba vytvořit interpretovaný scriptovací jazyk pro zlepšení interaktivity webových stránek, který by měl jít vložit přímo do HTML. Tímto úkolem byl pověřen Brendan Eich a jelikož bylo z ekonomických důvodů potřeba práci vykonat rychle, první funkční verze JavaScriptu byla údajně zhotovena za pouhých 10 dní\cite{the-origin-of-javascript}.

  Dnes se jedná o webový standard, díky kterému přidat to webové stránky interaktivitu a o dlouhodobě nejpoužívanější programovací jazyk na platformě GitHub\cite{github-statistics} a na kterém stojí řada podstatných funkcí tohoto projektu.


  \subsubsection{p5.js} \label{sec:p5.js}
  p5.js je JavaScriptová knihovna založena na softwaru pro tvorbu interaktivních „skečů“ Processing. Umožňuje vykreslovat tvary a křivky pomocí volání jednoduchých funkcí, proto se jedná o populární volbu pro umělce, učitele, či designéry.


  \subsubsection{\texorpdfstring{\KaTeX}{KaTeX}} \label{sec:KaTeX}
  \KaTeX{} (\url{https://katex.org/}) je JavaScriptová knihovna vytvořena pro vzdělávací stránku Khan Academy, která je na stránce používána na renderování matematických rovnic z \LaTeX ové notace (viz. kapitola \ref{sec:Matematika}).

  Populární alternativa, která byla na stránce dříve používána je knihovna MathJax (\url{https://www.mathjax.org/}), která je však oproti \KaTeX u znatelně pomalejší, proto jí byla nahrazena\cite{katex-mathjax-comparison}.


  \subsection{HTML} \label{sec:HTML}

  \subsection{CSS} \label{sec:CSS}

  \subsection{Python} \label{sec:Python}
  Python je programovací jazyk vytvořený pro distribuovaný operační systém Amoeba Guidem van Rossumem v 80. letech 20. století. Byl silně inspirován Guidovou prací na jazyce ABC -- samotný Rossum o Pythonu prohlásil: „Vzal jsem ingredience z ABC a trochu jsem je promíchal. Python je v mnoha věcech podobný ABC, ale v mnoha také rozdílný.\cite{making-of-python}“

  Od svého vzniku se Python vyvinul v mocný a frekventovaně používaný programovací jazyk -- má silnou standardní knihovnu, podporuje mnoho různých programovacích paradigmat a je třetím nejpoužívanějším programovacím jazykem na platformě GitHub\cite{github-statistics}, od stavby webových stránek po knihovny na trénování neuronových sítí.

  Na stránce zastává funkci scriptovacího jazyka pro automatizaci provozu (viz. kapitola \ref{sec:Automatizace provozu}).


  \subsection{Markdown} \label{sec:Markdown}
  Markdown je styl formátování prostého textu, vytvořený v roce 2004 Johnem Gruberem pro autory webových stránek. Je uzpůsobený k jednoduchému čtení, psaní, a převodu do pokročilejších značkovacích jazyků jako HTML\cite{markdown-history}.

  V projektu je v kombinaci s šablonovým jazykem Liquid používán ke psaní článků (viz. kapitola \ref{sec:Tvorba článků}).


  \subsection{\TeX} \label{sec:TeX}
  \TeX{} je programovací jazyk pro sázení odborné literatury vytvořený Donaldem E. Knuthem v 70. létech 20. století. Počáteční impulz pro vznik byla Knuthova nespokojenost s tiskovou kvalitou při vydávání jedné z jeho knih, což jej vedlo ke studiu principů sázení, tvorby fontů a později až k vytvoření vlastního sázecího systému \TeX\cite{tex-history}.

  Od klasických WYSIWYG (what you see is what you get) programů pro tvorbu dokumentů jako MS Word či LibreOffice se liší tím, že autor při psaní nevidí, jak dokument vypadá. Namísto toho pomocí „maker” (souborů instrukcí) definuje, jak by dokument vypadat měl a \TeX{} se o samotné sázení postárá. Tento přístup klade větší důraz na obsah -- autor se formátováním do značné míry nemusí zabývat a může se soustředit na psaní. Další výhoda \TeX u je fakt, že celý dokument je prostý text, proto jej lze verzovat.

  V projektu je pro převod stránky do offline verze používán \LaTeX{} (viz. kapitola \ref{sec:Převod webové stránky do PDF}). Jedná se o nadstavbu \TeX u, která díky pokročilejším makrům jako automatické číslování stránek, vkládání referencí, aj. umožňuje pohodlnější tvorbu dokumentů\cite{getting-started-with-latex}. K převodu \LaTeX u do formátu PDF je používán český program pdf\TeX{}\cite{pdftex}.


  \section{Použité nástroje} \label{sec:Použité nástroje}
  Tato kapitola rozebírá v projektu použité nástroje, jejich účel a porovnání s alternativami.

  Hlavní kritérium výběru byla multiplatformnost, jelikož jsou při vývoji aktivně využívaný oba operační systémy Windows a Linux.


  \subsection{Git} \label{sec:Git}
  Git je otevřený\footnote{Zdrojový kód Gitu je dostupný na adrese \url{https://github.com/git/git}.} verzovací systém vytvořený Linusem Torvaldsem pro použití na operačním systému Linux. V open-source komunitě se v současné době jedná o nejpoužívanější vezrovací systém\cite{version-control-usage-statistics}, což je jeden z hlavních důvodů pro jeho využití v tomto projektu.

  Verzovací systém umožňuje zaznamenávat změny (a jejich důvod/význam) na verzované skupině souborů a složek (tzv. repozitáři), aby se zamezilo ztrátě práce, zjednodušilo se hledání chyb v kódu a usnadnila se spolupráce mezi autory.

  Narozdíl od centralizovaných verzovacích systémů jako Subversion a CVS je Git decentralizovaný -- místo jedné centrální úschovny kódu má každý svou lokální kopii s úplnou historií projektu. Výhody tohoto přístupu oproti centrální úschovně jsou např. rychlost práce s repozitářem, možnost pracovat na projektu bez internetového připojení a zamezení ztráty kódu při selhání centrálního serveru\cite{cvcs-vs-dvcs}.

  To, že má každý svou lokální kopii však nezamezuje existenci hlavní kopie. Tento fakt dokazují služby jako GitHub, GitLab či GitBucket, které tuto centrální úschovnu poskytují.


  \subsection{Jekyll} \label{sec:Jekyll}
  Jekyll je otevřený\footnote{Zdrojový kód Inkscapu je dostupný na adrese \url{https://gitlab.com/inkscape/inkscape}.} generátor statických webových stránek napsaný v jazyce Ruby. Od tradičních přístupů ke stavbě webové stránky využívajících databáze či systém pro správu obsahu se liší tím, že stránku generuje pouze z textových souborů udávajících obsah a vzhled stránky. Výsledný produkt je plně statická stránka, která není napojena na žádný dynamický systém.

  Tento přístup výrazně zjednodušuje verzování, jelikož je převážná většina souborů projektu prostý text. Další výhoda je ochrana před potenciálními útoky, protože u statických webových stránek není řada frekventovaně používaných útoků možná -- produkt si neukládá nic, co by útočník mohl zneužít.

  Nejpopulárnější alternativna je Hugo, který byl při vývoji stránky rovněž testován, avšak nakonec byl kvůli aktivnější komunitě a osobní preferenci zvolen Jekyll.


  \subsection{Inkscape} \label{sec:Inkscape}
  Inkscape je otevřený\footnote{Zdrojový kód Jekyllu je dostupný na adrese \url{https://github.com/jekyll/jekyll}.} multiplaformní software na tvorbu vektorové grafiky. Oproti rastrovým softwarům, které pracují se samotnými pixely, si vektorové pamatují informace o tom, z jakých tvarů se obrázek skládá, což z nich dělá ideální kandidáty pro tvorbu ilustrací (viz. kapitola \ref{sec:Ilustrace}).

  Jedná se o populární volbu pro ty, kteří nepoužívají profesionální software jako Adobe Illustrator z důvodů ceny, uzavřenosti kódu či nedostupnosti pro operační systém Linux.


  \subsection{Fusion 360} \label{sec:Fusion 360}
  V rámci tvorby ilustrací k odborné literatuře je kromě tradičních přístupů rozebíraných v kapitole používaný CAD (computer-aided design) software, který slouží ke tvorbě technických nákresů. Na rozdíl od tradičních programů na tvorbu grafiky jsou rozměry tvořených objektů přesně definovány v jednotkách SI, aby odpovídaly objektům reálného světa.

  Mezi členy FRC teamů je populární volba CAD softwaru Fusion 360 od společnosti Autodesk, převážně protože je licence pro studenty (a jejich mentory) zdarma. Profesionální alternativou je program SolidWorks, který je často používán studenty technických škol.

  Jelikož za Fusion 360 neexistuje kvalitní multiplatformní náhrada, tak se jedná o jediný v projektu použitý program, který nefunguje na operačním systému Linux.

  \section{Vývoj webové stránky}
  % Rozhodl jsem se pro stránku
  % OS
  % Multiplatformn
  % Ilustrace fungování jak na mobilu, tak na PC?

  \subsection{Konfigurace Jekyllu}
  % Hledání themu.
  % Konfigurace themu.

  \subsection{Webhosting a doména} \label{sec:Webhosting a doména}
  Na českém trhu operuje velké množství firem, které zajišťují jak registraci domény, tak hostování samotné stránky. Každá má své výhody a nevýhody, proto je volba provozovatele vysoce individuální a záleží na řadě faktorů.

  Stránka je hostována a doména zprostředkována společností WEDOS Internet a její službou NoLimit (\url{https://hosting.wedos.com/cs/webhosting.html}), zejména kvůli mým pozitivním zkušenostem s touto službou v rámci minulých projektů a faktem, že ceny a parametry služeb jiných hostingů se od využívané příliš neliší.

  Jelikož je Jekyll podporován GitHubem, na kterém je hostován zdrojový kód stránky, další možností je služba GitHub Pages (\url{https://pages.github.com/}). Stránka by byla hostována zdarma, nebylo by třeba obsah nahrávat přes FTP a generování stránky by probíhalo automaticky. Tento přístup by však znemožnil využívání scriptů, proto není pro účely projektu vhodný.


  \subsection{Design loga}
  Na dobré vizáži každé stránky má značný podíl její logo, které by mělo tematicky odpovídat zaměření stránky bez toho, aby působilo příliš složitě.

  Hlavní nápad za vizáží loga je spojení výrazů „RO“ a „SI“ (první dva znaky jména projektu) do jednoho. Toto spojení výrazy zjednodušuje a odpovídá tak tematicky obsahu webové stránky, která je založena na jednoduchosti a intuitivnosti.

  Tvar znaku „O“ je rovněž podobný ozubenému kolu, což do designu loga pro stránku zaměřenou na robotiku elegantně zapadá.

  \begin{figure}[H]
    \minipage{0.95\textwidth}
      \includegraphics[width=\linewidth]{logo.png}
      \caption{Vývoj loga stránky} \label{img:Vývoj loga stránky}
    \endminipage
  \end{figure}

  Obrázek č.~\ref{img:Vývoj loga stránky} znázorňuje proces tvorby loga: počáteční nápad, následné přidání ozubeného kola a drobné úpravy na závěr.

  K designu loga byl použit program na úpravu vektorové grafiky Inkscape (viz. kapitola \ref{sec:Inkscape}).


  \section{Obsah webové stránky}

  \subsection{Cílové publikum}
  Jako cílové publikum jsem si vybral lidi v podobné situaci, ve které jsem byl já v roce 2017 -- jedinci se zájmem o robotiku, kteří jsou frustrovaní nedostatkem kvalitních materiálů pro studium. Materiál není kromě zájmu o robotiku omezen věkem, pohlavím či jinými osobními charakteristikami.


  \subsection{Tvorba článků} \label{sec:Tvorba článků}
  % Jak probíhá psaní článků.
  % Jak vypadá struktura článku - přirovnání k Wikipedii.

  \subsection{Struktura obsahu}

  \section{Funkčnost webové stránky}
  Následující kapitola pokrývá některé ze zajímavějších technických řešení použitých při tvorbě projektu.


  \subsection{Automatizace provozu} \label{sec:Automatizace provozu}
  Ke tvorbě kvalitního vzdělávacího materiálu je potřeba plynulý a do co největší míry automatizovaný provoz stránky -- přidávání článků by mělo být pouze vytvoření nového souboru a jeho umístění do příslušné složky projektu.

  Stránka je v tomto duchu automatizována scripty psanými v jazyce Python (viz. kapitola \ref{sec:Python}), aby se autor mohl plně soustředit na práci.


  \subsubsection{Nahrání obsahu přes FTP}
  FTP (File Transfer Protocol) je protokol zprostředkovávající přenos souborů mezi počítači po síti. Jedná se o populární volbu protokolu pro hostingy webových stránek a hosting tohoto projektu není výjimkou.

  Script \texttt{upload.py} se po zadání hesla připojí přes protokol FTP na server, rekurzivně smaže soubory a adresáře aktuální verze stránky a nahraje verzi novou. Pro dodatečné zabezpečení je IP serveru šifrována symetrickou šifrou AES.


  \subsubsection{Generování souboru sitemap.xml}
  Protokol sitemap (\url{https://www.sitemaps.org/protocol.html}) udává informace o pořadí procházení, časech změny a relativní prioritě částí stránky. Slouží vyhledávacím portálům, které tyto informace využívají pro inteligentnější indexování stránky ve svém systému.

  Script \texttt{sitemap.py} podle pořadí článků generuje záznamy do souboru \texttt{sitemap.xml}. Informace o umístění a pslední úpravě jsou získávány z atributů souboru a priorita je přidělena podle pozice na stránce -- úvodní stránka má prioritu $1.0$, hlavní články mají prioritu $0.8$ a vedlejší články prioritu $0.6$.


  \subsubsection{Převod webové stránky do PDF} \label{sec:Převod webové stránky do PDF}
  Pro offline dostupnost existuje mnoho různých typů souborů, na které by stránka šla převést. Často používané formátu jako \texttt{doc} a \texttt{docx} jsou však pro naše použití nevhodné, protože se mohou na různých zařízení zobrazit jinak. Je tedy vhodnější použít PostScript či PDF (Portable Document Format), jejichž vizáž na prostředí závislá není\cite{history-of-pdf}.

  Přímočará varianta by tedy byla převést za sebe seřazené články programem pro převod z formátu Markdown do formátu PDF (např. Pandoc). Tímto přístupem by však bylo obtížné měnit úrovně sekce podle toho, jedná-li se o hlavní či vedlejší články, generovat obsah a úvodní stránkum a prakticky nemožné upravit výsledné formátování.

  Pro potřeby projektu byl tedy zvolen převod formátu Markdown do \LaTeX u (viz. kapitola \ref{sec:TeX}) a až poté do formátu PDF. Tento přístup elegantně řeší všechny výše zmíněné nevýhody Pandocu.

  Script \texttt{tex.py} získá pořadí článků, spojí je za sebe do jednoho dokumentu a poté na ně aplikuje řadu regulárních výrazů, které provedou konverzi z formátu Markdown do formátu \LaTeX.

  % příklad konverze


  \subsubsection{Komprimace obrázků}
  O komprimaci obrázků pro zmenšení velikosti stránky se stará script \texttt{compress.py}, který pomocí služby TinyPNG (\url{https://tinypng.com/}) a jejího Python API zkomprimuje všechny obrázky vygenerované stránky. Samotný API klíč je pro zamezení zneužití šifrován symetrickou šifrou AES.

  Úprava fotek funguje na principu \emph{kvantování barev} -- podobné barvy jsou spojeny do jedné, proto jdou z tradičních 24-bitových palet PNG obrázků udělat palety 8-bitové, čímž lze obrázek zmenšit bez výrazného zhoršení kvality.


  \subsubsection{Minimalizace zdrojového kódu}
  Zkompaktnění zdrojového kódu je další možná optimalizace, kterou lze načítání stránky urychlit. Prohlížečům na vzhledu kódu nezáleží a běžné uživatele nezajímá, proto jej lze na úkor čitelnosti zmenšit.

  Tuto funkci zastává script \texttt{minify.py}, který soubory typu CSS a HTML zmenší odstraněním komentářů, nepotřebných tagů a uvozovek, přebytečných mezer a úpravou dalších věcí, které funkcionalitu kódu nezmění.


  \subsubsection{Automatizace procesu}
  Po přidání či úpravě článku je potřeba verzi stránky na hostingu aktualizovat. Tento proces řídí script \texttt{deploy.py}, který stránku nejprve pomocí Jekyllu vygeneruje, poté ve správném pořadí spustí všechny scripty pro generování dodatečného obsahu a optimalizace, a nakonec i script pro nahrání stránky na webhosting.


  \subsection{Verzování zdrojového kódu}
  Stránka je verzována pomocí verzovacího systému Git (viz. kapitola \ref{sec:Git}). Kód je otevřený a volně dostupný přes GitHub na adrese \url{https://github.com/xiaoxiae/Robotics-Simplified-Website}.


  \section{Analytika a propagace webové stránky}
  % Jak probíhá provoz serveru.

  \subsection{Analýza návštěvnosti}

  \subsubsection{Google Analytics}
  % Na co to je
  % Proč Analytics
  % Ukázat hezké statistiky a grafy

  \subsubsection{Google Search Console}
  % Na co to je
  % Proč Search
  % Ukázat hezké statistiky, grafy

  \subsection{Marketting}

  \subsubsection{Sociální média}
  % Reddit, Hacker News, Facebook...

  \subsubsection{FRC a FLL teamy}

  \subsection{Testování výkonu}
  % https://www.webpagetest.org/
  % https://sitechecker.pro/seo-report/

  \subsubsection{Rychlost načítání}
  % Kolik dat stránka načítá.
  % Jak rychle se načítá z různých zemí světa.

  \subsubsection{Multiplatformnost}
  % Jak to funguje na různých prohlížečích.
  % Jak to funguje na různých platformách.

  \section{Závěr}

  \newpage

  % create the bibliography
  \renewcommand{\refname}{Použitá literatura}
  \bibliographystyle{czechiso}
  \bibliography{soc}

  % create the list of figures
  \renewcommand{\listfigurename}{Seznam obrázků a tabulek}
  {%
  \let\oldnumberline\numberline%
  \renewcommand{\numberline}{\figurename~\oldnumberline}%
  \listoffigures%
  }

\end{document}
